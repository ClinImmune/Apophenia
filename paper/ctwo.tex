\chapter{\treesymbol More C tools}

If you have a good handle on the Chapter \ref{c_crash}, then you already
have what you need to write some very advanced programs. But C is a
world unto itself, with hundreds of utilities to facilitate better
coding and many features for the seasoned programmer.

This chapter covers some of the additional details of C and its
environment. You can read the rest of the book without reading this
chapter, but it discusses some features and tools that you may
eventually find to be useful.


\section{Syntactic sugar}   
There are several ways to do almost everything in Chapter \ref{c_crash}.  For
example, one could write the three lines \ci{b = (i > j); a += b;
i++;} as the single expression \ci{a += b = i++ > j;}. The seventh
element of the array \ci{a} can be called \ci{a[6]},
\ci{*(a + 6)}, or even (for the truly perverse) \ci{6[a]}.  Some people find one-liners and pointer arithmetic
to have \ae{}sthetic appeal, and there are even cases where they may
be clearer than the more common methods. But they are an easy way to
write impossible-to-debug code, and require mastery of a great number
of details of C internals that are wholly unnecessary otherwise.

The reader whose interest is piqued and would like to learn more about
how C works and about the many alternatives that I did not mention here
is referred to the authoritative and surprisingly readable reference
for the language: \cite{kandr:c}.

This optional section includes some of the more common alternatives that
are commonly used among C programmers. You can get very far without
knowing any of the following, and some of it is generally considered to
be bad style.

\paragraph{The obfuscatory if} There is another way to write an \cind{if} statement. The following are equivalent:
\begin{lstlisting}
if (a < b)
      a;
else
      b;
\end{lstlisting}
and
\begin{lstlisting}
(a < b) ? a : b;
\end{lstlisting}
Both have all three components: first the condition, then the `what to do if the
condition is true' part, and then the `what to do if the condition is false'
part. However, the first is more-or-less legible to anybody who knows basic English,
and the second takes the reader a second to parse every time he or she
sees it. Again, there are situations where the condensed form is useful,
but you could live a successful career writing in C without ever using it.


\paragraph{Macros} \label{macros} 

The other main use of the preprocessor is to define macros, which take
a bit of text and expand it to more text. The most common example is
expanding\\
\ci{MIN(a,b)}\\
to:\\
\ci{if (a $<$ b) a; else b;} .

But there are endless pitfalls to using macros, and since it is often
difficult to visualize how a complex macro will expand, debugging
a macro---or working out that it is the macro that is broken---is
difficult.  If you do write macros and need to debug them, the \ci{-E} flag to \ci{gcc} will run only the \ind{preprocessor}, so you can
see what expands to what.

Here, by the way, is the actual code for the \ci{GSL\_MIN} macro;
it follows the first rule of macro writing, which is that everything
should be in parentheses:\\
\comment{\ci{\#include <gsl/gsl\_math.h>\\
\#define GSL\_MIN(a,b) ((a) < (b) ? (a) : (b))} \cindex{GSL\_MIN}\\}
\begin{lstlisting}
#include <gsl/gsl_math.h>
#define GSL_MIN(a,b) ((a) < (b) ? (a) : (b)) 
\end{lstlisting}
\cindex{GSL\_MIN} \cind{GSL\_MAX} is similarly defined.


The one and only one thing that a macro can do better than a function is
take a type as an argument. This takes advantage of the fact that the preprocessor 
just shunts characters around and doesn't know what its arguments represent.
For example, recall the form for reallocating a pointer to an array of \ci{int}s:\\
\ci{var\_array = realloc(var\_array, new\_length * sizeof(int))}.\\
With this macro:
\begin{lstlisting}
#define REALLOC(ptr, length, type) ptr = realloc(ptr, length * sizeof(type))
\end{lstlisting}
the above line could be rewritten as:\\
\ci{REALLOC(var\_array, new\_length, int);}\\
This macro gives you one more moving part that could break (and which
now needs to be \ci{\#include}d with every file), but
may make the code more readable.
\comment{\ci{\#define REALLOC(ptr, length, type) ptr = realloc(ptr, length * sizeof(type))}\\
\ci{REALLOC(var\_array, new\_length, int);}\\}

Finally, notice that the custom is to put macro names in caps.  You can
rely on this in code you see from others, and are encouraged to stick
to this standard when writing your own.

\paragraph{Labels and \ci{break}} A line of code can be named, by simply providing a
name with a colon after it. You can then jump to that line via \cind{goto}. Here is a code snippet that presents the basic idea, with a line labeled \ci{outro}:
\begin{lstlisting}
for (i=0; i< vector_size; i++){
    if (a_vector[i] > b){
        out = a;
        goto outro;
    }
}
out = b;

outro:
free(a_vector);
free(another_vector);
return out;
\end{lstlisting}
The goto is reviled by modern computer scientists as terrible style, so
it appears infrequently. However, redundant code is also bad style. 
In the above example, getting rid of the \ci{goto} would require copying
the various memory-freeing tasks at the end of the function into the
center of the \ci{for} loop.  Linus Torvalds,\index{Torvalds, Linus}
the author of the Linux kernel, recommends the \ci{goto} to eliminate
redundancy as above; many other authors choose to never use \ci{goto}.

An alternative is \cind{break}, which cuts out of the innermost loop in
which it is located. It is primarily useful when searching through an
array for an item. Once you have found what you are looking for, there
is no need to continue looping to the end of the array.
Here is code that would work about like the above:
\begin{lstlisting}
for (i=0; i< vector_size; i++){
    if (a_vector[i] > b){
        out = a;
        break;
    }
}
if (out != a)
    out = b;
free(a_vector);
free(another_vector);
return out;
\end{lstlisting}

\paragraph{Switch} The \cind{switch} statement is a clean way to
branch among many options. First, you will need a variable indicating
a single character. For example, the \ci{get\_opt} function from the
standard library will parse command line arguments; then the \ci{switch}
command will branch depending on the value returned:
\begin{lstlisting}
char c;
c = get_opt(...);
switch(c){
   case 'v':
        verbose++
        break;
   case 'w':    
        weighting_function();
        break;
   case 'f':      
        fun_function();
        break;
}
\end{lstlisting}
So when 
\ci{c == 'v'}, the verbosity level is increased,
when \ci{c == 'w'}, the weighting function is called, 
et cetera.

Note well the abundance of \ci{break} statements.  The \ci{switch}
function just jumps to the appropriate label (recall that the colon
indicates a label) and then picks up from there---and continues. Thus,
if there were no \ci{break} after \ci{verbose++}, then the program
would merrily continue on to execute \ci{weighting\_function}, and so
on. This is called {\sl fall-through}. The
reader who uses \ci{switch} statements will want to take care
to have \ci{break}s at the end of every \ci{case}.  The \ci{break}
at the end of the list of \ci{case}s is extraneous, but there is a good
chance that you will add to your list of \ci{case}s, at which point
the \ci{break} will no longer be extraneous and will prevent a fall-through bug.

An alternative to the \ci{switch} is a simple series of \ci{if}s:
\begin{lstlisting}
char c = get_opt(...);
if (c == 'v'){
        verbose++
} else if (c == 'w'){
        weighting_function();
} else if (c == 'f'){
        fun_function();
}
\end{lstlisting}
There is more redundancy here---the name \ci{c} is repeated three times
where in the \ci{switch} setup it was used once---but there is no risk
of fall-through mistakes.


\paragraph{Optimization} \index{optimization}
The \cind{gcc} compiler can do a number of things to your code to make it
run faster. For example, if you assign \ci{a = b + c}, it may replace
every instance of \ci{a} with \ci{b + c}, or it may change the order
in which lines of code are executed. To turn on \ind{optimization},
use the \ci{-O3} flag when compiling with \ci{gcc}. [That's an `O'
as in optimization, not a zero. There is also an \ci{-O1} and an \ci{-O2}, but as long as you are optimizing, why not go all out?]

The problem with optimization, however, is that it makes debugging
difficult. The program jumps around, making stepping through an odd
trip, and when you ask to print the value of \ci{a}, you will get
an error that there is no such variable defined.
It also sometimes happens that you did not do your memory allocation duties
quite right, and things went OK without optimization, but suddenly the
program crashes when you have optimization on; the debugger will be some
help, but you may just have to re-scour your code to find the problem.
Thus, the \ci{-O3} flag is a final step, to be used only after you
are reasonably confident that your code is debugged.

Finally, you may run into authors who give you advice about how to write
faster code, telling you to use loops here and unroll them there, make
frequent use of the \cind{inline} keyword, or order your commands based
on memory usage instead of human logic. Ignore them. It is much more
valuable to minimize the time you spend deciphering and debugging your
code than shaving seconds off of your run time.  Write code to maximize
readability for yourself and your fellow humans. If you want optimization,
\ci{gcc -O3} will do almost all of these rearrangements for you.


\section{Function pointers} \index{function pointers}
A data point \ci{d} is stored somewhere in memory, so we can refer to
its address, \ci{\&d}. Similarly, a function \ci{f} is stored somewhere
in memory, so we can refer to its address as well.

What is a pointer to a function good for?  It lets us write
functions that will accept any function pointer and then use the
pointed-to function, such as a bootstrap function that takes in a
statistic-calculating function and a data set and then returns the
variance of the statistic, or a function to search an input function
for its largest value.

\subsection{Types} Before we can start writing functions to act on
functions, we need to take the type of input function into
consideration, because not every function makes sense in every context. Say
that we want to write a function that will take in an array of
\ci{double}s and a function, and will apply the function to every
element of the array. Then the input function has to have a form like:
\begin{lstlisting}
double function_name (double x);
\end{lstlisting}

If the function took in \ci{char*}s or \ci{gsl\_vector}s, it just
wouldn't work; nor would it work if it returned anything but a
\ci{double}.

Thus, function pointers have types just like any other variable, and a
function that takes a function pointer as input will make the usual
type-checks.

A pointer declaration is just like any other declaration but with
another star, like \ci{int *x}; the same goes for declaring function
pointers, but there are extra parens. Here is a declaration for a
function pointer that takes in a \ci{double} and returns a \ci{double}:
\begin{lstlisting}
double (*function_name) (double x);
\end{lstlisting}
You can see that it is just like the usual \ci{int *x} sort of pointer declaration, but the function is in parens.  The parentheses are not
optional.

\paragraph{\ci{typedef} revisited}
Now you can put the function into your header line. A function that
applies a function to a \ci{gsl\_vector} would have a header like
this:
\begin{lstlisting}
void apply (gsl_vector *v, double (*function_name) (double x));
\end{lstlisting}

Are you confused yet? Each component basically makes sense, but together
it is cluttered and confusing. There is a way out: \cind{typedef}. By prepending
that word before the above type declaration,
\begin{lstlisting}
typedef double (*double_to_double) (double x);
\end{lstlisting}
we have created an new type named \ci{double\_to\_double} that we can
use like any other type. Now, the declaration of the \ci{apply} function
simplifies to:
\begin{lstlisting}[emph={double_to_double,gsl_vector}]
void apply (gsl_vector *v, double_to_double function_name);
\end{lstlisting}

\subsection{Putting \ci{typedef} to work}
Figure \ref{plotafunction} shows a program to plot any function of the
form $\Re\to\Re$. The
details of Gnuplot and its use probably won't make sense to you until
you read Chapter \ref{gnuplot}, and you may need to modify the
\ci{gnuplot} variable on line 5 to indicate the location of Gnuplot on
your system.\footnote{try the command \bi{which gnuplot} to find it,
and if it isn't found, use your package manager to install it.}

That aside, the elements needed to pass function pointers are all there,
and should be clear to you. A quick tour of the code:\\
Line 3: To make life easier, the \ci{dfn} type is
declared at the top of the file. \\
Line 7: The \ci{sample\_function} appears, and you can see that its type matches that of \ci{dfn}. \\
Line 11: The \ci{plot\_a\_fn} function specifies that it takes in a function of type
\ci{dfn}.\\
Line 19: Using the passed-in function is as simple
as using any other function: this line  gives no
indication that \ci{plotme} is in any way special.  \\
Line 26: Finally, in \ci{main},
you can see how \ci{plot\_a\_fn} is called. The \ci{sample\_function}
is passed in with just its name.

Once the \ci{typedef} is in place, you can see that the syntax is pretty
easy. You don't need extra stars in either the declaration of the
function-of-a-function or in the call to that function, and you can call
the pointed-to function like any other. 

\lstset{numbers=left, numberstyle=\scshape}
\codefig{plotafunction}{A demonstration of a function that takes in
any function $\Re\to\Re$ and plots it.}
\lstset{numbers=none}

\exercise{Define a type \ci{dfn} as in line three of Figure \ref{plotafunction}. Then
write a function with header \ci{void apply(dfn fn,
double *array, int array\_len)}, that takes as arguments a function, an
array, and the length of the array, and changes each element
\ci{array[i]} to \ci{fn(array[i])}.

Test your function by creating an array of the natural number $1, 2,
3,\dots 20$ and transforming it to a list of squares.}

\subsection{\ci{void}ing types} Forthcoming.

\summary{\item You can pass functions as function arguments, just as you
would pass arrays or numbers.
\item The
syntax for declaring a function pointer is just like the syntax for
declaring a function, but the name is in parens and is prepended by a 
star.
\item Defining a new type to describe the function helps immensely.
This requires prepending \ci{typedef} to the function pointer
declaration in the last summary point.
\item Once you have a \ci{typedef} in place, you can declare functions
that take functions, use the passed-in functions, and call the parent
function as you would expect. Given the \ci{typedef} You need neither
stars nor ampersands for these operations.  
\item If the input function can take multiple types of input, use a
\ci{void *} to tell the compiler not to check types. Use this with
caution.
}

\section{Data structures} Forthcoming, covering linked lists versus trees versus arrays. Primarily of use to modelers.

\section{Parameters} \index{Command line|arguments on}Your simulations and analyses will require
tweaking. You will want to try more agents, or you may want your program
to load the data from a text file for one run and then use the data in
the database for later runs.

There are three places where you can put such parameters and
specifications. The first option is to put them at the top of your \ci{.c}
file or a header file. This is trivial to implement, but you will need
to recompile every time you change parameters.

The second option is to put the data in a parameter file. There are
various libraries that will read parameter files in various foramts that
you can find online, and once you have learned about how to read data
from a text file to a database in Chapter \ref{sql}, you will have
another easy means of reading in a text file of parameters. 
The implementation is fairly easy given the available tools, and it is fairly easy to change 
a text file, since you won't have to recompile after changing it.

The third option is to read in parameters from the command line. This
takes the most effort to implement, but allows you to change parameters
every time you run the program. You can even write batch files in Perl,
Python, or a shell-type language to run the program with different parameters variants.

The easiest way to read in command line parameters is via the
\ci{getopt} set of commands. They are part of the Standard C
library, so they are fully portable.\footnote{As an alternative, the
GNU C library provides a library named \cind{argp} that provides many
more features and does more automatically, but is correspondingly more
complex and less portable.}

\section{More tools} 
Beyond there debugger, there are a few more programs that will make your life as a programmer easier.

\subsection{Help} \index{help, getting}
You probably have have all the manuals for the programs listed in this
chapter on your hard drive now. Most of them are in \TeX info format,
which you can read by commands such as \bi{info gcc} or \bi{info make}. If
you have trouble navigating in the \bi{info} program, then you can
get help with \bi{info info}.

There are manual pages for most of the C functions in the standard
libraries. Try \bi{man printf} or \bi{man atoi}, for example.

This help may not be available on every system---it's a bit of
a crap shoot. But all of these documents are online. Just enter the
command you would have typed at the command line into your favorite
search engine. A search for \bi{info gsl} or \bi{man printf} will
turn up exactly the documentation that is missing from your system,
formatted for the web.

\subsection{Make} \label{make} \index{make@\bi{make}|(}
Once your program has many sections (or sooner), you will want to use
\bi{make} to automate the process of turning your mess of files into
a C program.  It is a program entirely separate from C, but which
is always found wherever compilers are.

Typically, you have ten \ci{.c} files, which will compile
to ten \ci{.o} files, which will then be linked into one executable.
You can describe the situation using a
text file named \ci{Makefile} (capital M). The file will specify a
series of dependencies: \ci{file1.o} depends on \ci{file1.c}, but does
not depend on \ci{file2.c}, while \ci{run\_me} depends on both {\tt
file1.o} and \ci{file2.o}.  Then, it specifies what actions need to be taken
when a file is dependent on something which has changed.  For example,
Figure \ref{makefig} is a Makefile for a program named \ci{run\_me}, which has two {\tt
.c} files and one header file.

\begin{figure*}
\begin{verbatim}
OBJECTS = file1.o file2.o           #User-defined
PROGNAME = run_me                   #User-defined
CFLAGS = -g -Wall
LINKFLAGS = -L/usr/local/lib -lgsl -lgslcblas -lsqlite
COMPILE   = gcc $(CFLAGS) -c $< -o $@

$(PROGNAME): $(OBJECTS)
        gcc $(CFLAGS) $(OBJECTS) $(LINKFLAGS) -o $(PROGNAME)

file1.o: file1.c my_headers.h       #User-defined
        $(COMPILE)
file2.o: file2.c my_headers.h       #User-defined
        $(COMPILE)
\end{verbatim}
\caption{A sample Makefile for a program with two source files.}
\label{makefig}
\end{figure*}

The white space at the beginning of the indented lines is a single tab,
not a bunch of spaces.  C never cares about this, but \bi{make}
does.

\paragraph{The executive summary} This Makefile is reprinted from the \airq{Setup}
section of the Apophenia documentation, at
\url{http://apophenia.info/doc}, and in the sample code at 
\samplecodelocation. Cut and paste it into a file named
\bi{Makefile}, taking care to retain the tabs, and put that file in
the same directory as your source code. Then, change the first two lines
to include one object file for each of your source files (you may only
have one), and
to the program name you prefer.  You may also need to change the
\bi{LINKFLAGS} line if you get errors from the linker about symbols
not found.

Having saved those changes, just run
\bi{make} from the command prompt to compile.
Also, many programs will let you run
make from within the program. \bi{gdb} lets you do this,
as will most implementations of \ind{vi} and \ind{EMACS}.


\paragraph{The details} For the curious and those who need to modify
the standard Makefile, here are some details about its format.

At the top are a list of variables, such as the list of object file names, the final program, et cetera.
When \bi{make} encounters \ci{\$(PROGNAME)}, it will insert the value of the variable \ci{PROGNAME} in that
slot, so the line 
\begin{lstlisting}
gcc $(CFLAGS) $(OBJECTS) $(LINKFLAGS) -o $(PROGNAME)
\end{lstlisting}
will translate into
\begin{lstlisting}
gcc  -g -Wall file1.o file2.o -L/usr/local/lib -lgsl -lgslcblas -lsqlite -o run_me
\end{lstlisting}
which looks an awful lot like what we had typed before (but not quite---this does the linking step only).

The remainder of the file are dependency lists and commands. Begin
with the second set: \bi{file1.o} depends on \bi{file1.c} and \bi{my\_headers.h}, meaning that if either file is newer than \bi{file1.o}
(as indicated by the time stamp), then you will need to recompile \bi{file1.o}. \bi{make} will do this by using the command which begins with
\bi{gcc} in the subsequent line. No need to go in to the details, but
this line will expand to \bi{gcc -g -Wall -c file1.c -o file1.o}. This
is the compilation step, without the linking. \bi{file2.o} has similar
dependencies.

The first dependency set says that \bi{\$(PROGNAME)}---that is, \bi{run\_me}---depends on all of the object files. Meanwhile, the object files
depend on the \bi{.c} files and \bi{my\_header.h}, as above. So when
you edit \bi{file1.c} and then type \bi{make} at the command prompt,
\bi{make} will see that \bi{file1.o} is now out of date and needs
recompilation. This change cascades: \bi{run\_me} is now out of date,
and needs to be re-linked.

Notice that \bi{file2.o} did not get recompiled, since it is still
up-to-date. Thus, not only will \bi{make} save you typing, it will also
save you time, since you will usually be hacking at one file at a time,
so there is no point waiting for the other object files to recompile
to exactly what they were before. 
\index{make@\bi{make}|)}

\subsection{Memory debugger} \index{memory debugger|see{valgrind}} \index{Valgrind|(}
\index{segmentation fault}

The setup is this: you make a mistake in memory handling early in the
program, but it is not fatal, so the program continues along using bad
data. Later on in the program, you do something innocuous with your bad
data and get a segfault. This is a pain to trace using \bi{gdb}, so
there are packages designed to handle just this problem.

The first alternative is \ind{Electric Fence}, a
library by Bruce Perens that allocates memory so that all memory
mis-allocations and mis-reads will immediately crash.  That means that
the debugger will stop on the line with the error, instead of continuing
along with its bad data, so your debugger will give you much more useful
information.  Electric Fence works by redefining \cind{malloc} in 
its library. To use it, you would recompile the function
using the efence library (adding \bi{-lefence} to the compilation
command).


Another option is \ind{Valgrind}, a program that will
run your program in anal-retentive mode, and check every memory operation. 
It is not available on all systems (it is generally Linux-centric), but
if it is available, it is easier to use (no recompilation) and provides much more information than efence or
\bi{gdb} alone. If anything breaks the rules, Valgrind will give you
the location, in the form of a backtrace very much like the backtraces
familiar from \bi{gdb}. It can even be set to start the debugger as soon
as it catches something wrong.\footnote{Use the \bi{db-attach} switch.
E.g., \bi{valgrind -v --db-attach=yes ./my\_program}.}

The usage is simple: \bi{valgrind my\_program}. If you have a prodigious amount
of errors, you may want to tack on the \bi{--logfile=problems} option. It doesn't
quite print to the file name you give; do a directory listing after you run \bi{valgrind} to see where it goes.  Your final option is to pipe the standard
error to a file: in the bash shell, which you are probably using: \bi{valgrind my\_program 2$>$ problems}.
\index{Valgrind|)}

\subsection{Revision control} \index{revision control|see{subversion}} \index{CVS|see{subversion}}\index{subversion|(}\label{valgrind}
The idea behind the revision control system (RCS) is that your project
lives in a repository. When you want to work, you check out
a copy of the project, and when you are done making changes, you check
them back in to the repository and can delete the copy.  The repository
makes a note of every change you made, so you can check out a copy of
your program as it looked three weeks ago as easily as you could check
out a current copy.

This has pleasant psychological benefits. Don't worry about experimenting
with your code: it is just a copy, and if you break it you can always check
out a fresh copy from the repository. Also, nothing matches the confidence
one gets from making major changes to the code and finding that the
results still match the results from last month to four decimal places.

Finally, revision control packages facilitate collaboration with
coauthors. If your changes are sufficiently far apart (e.g., you are
working on one function and your coauthor on another in the same file),
then the RCS will merge all changes to a single working copy. If
it is unable to work out how to do so, then it will give you a
clearly demarcated list of changes for you to accept or reject.

This method also works for any other text files you have in your
life, such as papers written in \LaTeX, HTML, or any other text-based
format. For example, this book is under revision control.

The standard revision control software (at the moment) is named
subversion; if it is not already installed, it is probably available as
a package for the system you are using. For usage, the reader is referred
to subversion's own detailed manual describing set-up and operation from
the command line.

If subversion is not readily available, the next best is its predecessor,
CVS, which is well-supported and even has a number of graphical
front-ends, such as tkCVS.  
\index{subversion|)}

\subsection{The \ind{profiler}} The profiler times how long every function
takes to execute. Its usage is much like the debugger. First, you need
to add a flag to the compilation to include profiler symbols,
\ci{-pg}. Then, execute your program, which will produce a file named
\bi{gmon.out} in the directory, with the timings that the profiler
will use. Unlike the debugger's \ci{-g} option, the \ci{-pg}
option may slow down the program significantly as it writes to
\bi{gmon.out}, so use \ci{-g} always and \ci{-pg} only
when necessary.

Finally, call \bi{gprof ./my\_executable} to produce a
human-readable table from \ci{gmon.\-out}.\footnote{\bi{gprof}
outputs to \ind{STDOUT}; use the usual shell tricks to manipulate
the output, such as piping output through a pager, \bi{gprof
./my\_executable | less}, or dumping it to a text file, \bi{gprof
./my\_executable > outfile}, that you can view in your text editor.}
See the manual (\bi{man gprof}) for further details about reading
the output.

As with the debugger, once the profiler points out where the most time
is being taken by your program, what you need to do to alleviate the
bottleneck becomes very obvious.

If you are just trying to get your programs to run, optimizing for speed 
may seem far from your mind. But it may be an interesting exercise to
run a modestly complex program through the profiler, because, like the
debugger's backtrace, its output provides another useful view of how
functions call each other.

\exercise{At this point you have a few programs with two or more
functions. [The profiler gets confused if your program only has a
\ci{main} and no subfunctions.] Run the profiler as above and have a
look at the output. Are the calls as you had expected? If you have
several assertions (you should), try recompiling with -DNDEBUG and see
how the profiler timings change.}



