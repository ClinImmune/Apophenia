%@-     so lgrind will play nice with makeindex
\chapter[Gaussian tricks]{Hypothesis testing with Gaussian distributions} \label{gauss}

This subsection covers most of what we traditionally learn in first-year statistics. 
\comment{
Much of it is obsolete. It covers parameters which we know to have a Normal (aka Gaussian) distribution,
or those things which are derived therefrom. These are nice because the math has already been done, by
people who lived before computers, meaning that we don't need metaphorically heavy machinery to do most of
the calculations here.} Most of the work will consist of taking a dot product, maybe inverting a matrix,
and then looking up a number in a table. 

Everything here depends on the Central Limit Theorem. If your data doesn't
fit the CLT, then please don't use these techniques. Work out how your
data is distributed, to the best of your abilities (try bootstrapping,
Chapter \ref{boot}), and then write down a likelihood function. If you're
looking to estimate model parameters, do a maximum likelihood estimation;
if you're looking to test a hypothesis, write down a likelihood ratio
based on the distribution you've just calculated.



\section{Meet the Gaussian family} \label{dist2}
With the exception of the Normal, the distributions below are distinct
from the distributions above. The distributions above are typically used
to describe data that we have observed in the real world. But having
gathered data, we often model the data using techniques such as
linear regressions, that produce means of model parameters, model
variances, and other parameters. Those model parameters have their own
distributions, and those are typically one of the distributions below.


\paragraph{The \ind{Central Limit Theorem}} is the first piece of magic
that we need. As $n\to \infty$, we know that $\sqrt{n} (\overline X -
\mu)/\sigma \sim {\cal N}(0,1)$, regardless of where $X$ came
from. With that regularity of nature, we can derive all of the following distributions.
\label{CLT}

\paragraph{Normal\index{Normal distribution}} 

The Normal distribution, defined above on page \pageref{normal},
will also be used to describe some of the parameters derived below.
The big problem with tne Normal distribution is that it depends on $\sigma$, an
unknown. It also depends on $\mu$, but if the statistic
being tested is unbiased---like most of the estimators of $\beta$ in the
catalog below---then you have $\mu$. Thus, much of the trickery in the
section on test design below involves combining distributions in ways
such that the unknown $\sigma$s cancel out.

\begin{itemize}
\item $\{{\cal N} (a,b)\} + \{{\cal N} (c,d)\} \sim {\cal N}
(a+c,b+d)$ (where $\{x\}$ is a variable with the given distribution).
\end{itemize}

\paragraph{$\chi^2$}\index{chi squared distribution@$\chi^2$ distribution} $[\sum_n (x-\overline
x)^2]/\sigma^2\sim \chi^2_{n-1}$. The numerator there is the sample
variance times $n$, so we can test that the sample variance equals a
given $\sigma^2$ with this. But unless we assume $\sigma$ like that,
we're up the same creek as with the Normal.

\begin{itemize}
\item $\{{\cal N} (0,1)\}^2 \sim \chi^2_1$

\item $\{\chi^2_1\}$ + \dots + ($n$ of these) + \dots + $\{\chi^2_1\} \sim \chi^2_n$

\end{itemize}			\label{chisq}

Notice that the sample variance at the top of this section is $\sim
\chi^2_{n-1}$, not $n$, because given the first $n-1$ data points, the
last one can actually be calculated from that data; meaning that we
effectively have the sum of $n-1$ $\{\chi^2_1\}$s plus a constant.

\paragraph{Student's t\index{t distribution}}  ${\overline X - \mu \over S/\sqrt{n}}\sim
t_{n-1}$, where $s=\uv'\uv/n$. This is a work of genius by Mr. Student.
By dividing a normal by $\sqrt{\chi^2_{n-1}/(n-1)}$, the unknown
$\sigma$s cancel, and we have a function whose elements are all known,
and its distribution.  If $n=2$, we call it a Cauchy distribution.
\label{tstat}


\paragraph{$F$\index{F distribution}}  This is the same cancel-the-$\sigma$s trick as with the $t$, but here the form
is $[\chi^2_m/m]/[\chi^2_n/n]\sim F(m,n)$. For example, if the numerator
is a ${\cal N}(0,1)$, and the denominator is a $\chi^2$ based on the
sample variance, then this is the square of the $t$ statistic.
The nice thing about the $t$ distribution is that it's easy and it
allows one-tailed hypothesis tests. But if you want to test a
multivalued hypothesis, then you need to do an $F$ test.

\subsection{Calling these functions from the GSL}
There are four things that cover most of what you will be doing. 
The first is simply looking up the vaule of the PDF at a given point. Here are the functions:
%\begin{verbatim}
\begin{lstlisting}
#include <gsl/gsl_randist.h>
double gsl_ran_gaussian_pdf (double X, double SIGMA);
double gsl_ran_tdist_pdf (double X, double NU);
double gsl_ran_fdist_pdf (double X, double NU1, double NU2);
double gsl_ran_chisq_pdf (double X, double NU);
\end{lstlisting}
%\end{verbatim}
The prefix \cinline{gsl\_ran} indicates that these functions are from the
random number generation module. (Random number generation itself will be delayed to page
\pageref{randomnumbers}.) Notice that there is no mean for the Normal and neither mean nor variance
for the T-distribution, so we may need to modify $X$ accordingly. For
example, here is a function for a t-distribution with approaches the
mean $M$ and variance $v$ as \cinline{NU}$\to \infty$:
%\begin{verbatim}
\begin{lstlisting}
#include <gsl/gsl_randist.h>
#include <math.h>      //sqrt
double gsl_ran_tdist_pdf_with_mv(double X, double mean,
                               double var, double NU){
    double X_scaled = (x - mean)/sqrt(var);
    return gsl_ran_tdist_pdf(X_scaled, NU);
}
\end{lstlisting}
%\end{verbatim}

Next, there's calculating the CDF above or below a point. 
The \cinline{P-}functions below calculate the CDF below a point, i.e.
$\int_{-\infty}^X f(y) dy,$
while the \cinline{Q-}functions calculate the CDF above a point, i.e.
$\int^{\infty}_X f(y) dy.$
These will obviously add to one, so you can express any area in terms of whichever function is clearest.

For example, if we find that our Normally-distributed mean is 2.2 standard
deviations above zero, then we can reject the one-tailed hypothesis that
the mean is zero with probability \cinline{1 - gsl\_cdf\_gaussian\_Q(2.2, 1)}.

\index{Gaussian distribution|see{Normal distribution}}
\index{chi squared distribution@$\chi^2$ distribution!gsl\_cdf\_chisq\_P@\cinline{gsl\_cdf\_chisq\_P}}
\index{t distribution!gsl\_cdf\_tdist\_P@\cinline{gsl\_cdf\_tdist\_P}}
\index{F distribution!gsl\_cdf\_fdist\_P@\cinline{gsl\_cdf\_fdist\_P}}
\index{Normal distribution!gsl\_cdf\_gaussian\_P@\cinline{gsl\_cdf\_gaussian\_P}}
\ttindex{gsl\_cdf\_...}
%\begin{verbatim}
\begin{lstlisting}
#include <gsl/gsl_cdf.h>
double gsl_cdf_gaussian_P (double X, double SIGMA);
double gsl_cdf_tdist_P (double X, double NU);
double gsl_cdf_fdist_P (double X, double NU1, double NU2);
double gsl_cdf_chisq_P (double X, double NU);
\end{lstlisting}
%\end{verbatim}
\dots plus all of these with the \cinline{P} replaced by a \cinline{Q}.


In the other direction, we may want to know where we will need to be to reject a hypothesis with 95\%
certainty. For example, a Value-at-risk oriented regulator will want to know what a bank is likely to lose 
one day in the month. That is, what is the value of the 1-in-20, or 5\%, point on the CDF?
Assuming a Normal(M,S) distribution of profit and loss,\footnote{This is false. Securities
typically have leptokurtic (fat-tailed) returns.} the bank will report a \index{value at risk} of {\tt
gsl\_cdf\_gaussian\_Pinv (0.05, S)+M}. Here are the other function delcarations:
\index{chi squared distribution@$\chi^2$ distribution!gsl\_cdf\_chisq\_Pinv@\cinline{gsl\_cdf\_chisq\_Pinv}}
\index{t distribution!gsl\_cdf\_tdist\_Pinv@\cinline{gsl\_cdf\_tdist\_Pinv}}
%\begin{verbatim}
\begin{lstlisting}
#include <gsl/gsl_cdf.h>
gsl_cdf_gaussian_Pinv (double X, double SIGMA);
double gsl_cdf_chisq_Pinv (double P, double NU);
double gsl_cdf_tdist_Pinv (double P, double NU);
\end{lstlisting}
%\end{verbatim}
\dots plus all of these with the \cinline{P}s replaced by \cinline{Q}s.
Notice that there is no inverse for the CDFs of the F distribution.





\section{Testing a hypothesis}




We have all the ingredients we need to test a hypothesis; all that
remains is to glue them all together. For example, in the last chapter, the code
found the coefficient vector \cinline{beta} and \cinline{xpx\_inv}, which is the
variance-covariance matrix of \cinline{beta}.  Now assume that \cinline{beta}
is normally distributed (for example, the original data set consists
of iid draws from a homoskedatic distribution). Then we can use the
above functions to find the probability that each element of \cinline{beta}
is different from zero:

%\begin{verbatim}
\begin{lstlisting}
\\We enter with the vector beta and matrix xpx_inv
#include <gsl/gsl_cdf.h>  //CDF functions
#include <math.h>         //abs
int i;
double confidence[beta->size];
for (i=0;i< beta->size; i++){
    confidence[i] = gsl_cdf_gaussian_Q( abs(gsl_vector_get(beta,i)), 
                         gsl_matrix_get(xpx_inv, i,i));
    confidence[i] -= .5;
    confidence[i] *= 2;
}
\end{lstlisting}
%\end{verbatim}
The last two steps converted from the one-tailed area below \cinline{beta}
to a two-tailed area between $\pm$\cinline{beta}.



\subsection{How to evaluate a test}
\marginalia{Evaluation vocab}{
Here are some vocabluary terms; if you are in a stats class right now,
you will be tested on this:

Likelihood of {\bf type I errors} $\equiv\alpha$: rejecting the null when it's true.\index{type I errors}

Likelihood of {\bf type II errors} $\equiv\beta$: accepting the null when
it's false.\index{type II errors}

{\bf Power}$\equiv 1-\beta$, or the likelihood of rejecting a false null.\index{power}

{\bf Unbiased}$\equiv (1-\beta)\geq \alpha\ \forall$ values of the parameter.
I.e., you are less likely to accept the null when it's false than when
it's true.\index{unbiased estimator}

{\bf Consistency}$\equiv$ the power $\to 1$ as $n\to \infty$.\index{consistency}
}

A test will either reject or accept a hypothesis given data. It can be
fooled two ways: the hypothesis could be true but the test rejects it;
or the hypothesis could be false but the test accepts it anyway. There
is a balance to be struck between the two errors: as one rises, the
other falls. But not all estimators are born equal: the `heads, accept;
tails, reject' testing method gives us a 50\% chance of a Type I error
and a 50\% chance of a Type II error, but there are tests where both
errors are significantly lower than 50\%, and so by any measure we would
call those better tests than the coin-flipping test. 


\subsubsection{Keeping type I and type II straight} 
[Somebody once told me this story: if a doctor is deciding
whether to give a vaccination, and $H_0=$ the patient is sick,
then rejecting $H_0$ when it's true and thus not vaccinating is much worse than accepting
$H_0$ when it's true and thus vaccinating extraneously, so the important error is type I. I always
thought this was totally {\it ad hoc}, especially since $H_0=$
the patient is not sick is more null. But if the story helps you, run
with it.]

You are probably familiar with the
standard hypothesis test, which says things like `we reject the null with 90\%
likelihood', and is what the charts in the back of the textbook are about.
That's the type I error. It comes first because
it's generally all anybody talks about.

Why do we neglect type II errors? Because in the next section Neyman
and Pearson will tell us we can.

\subsection{Likelihood ratios} \index{likelihood ratio}

Say the cost to a type I error is $C_I$ and the cost to a type II error is $C_{II}$.
Then you'll reject $H_0$ iff the likely cost is less than the likely cost of not
rejecting. That is reject $H_0$ iff $C_I P(H_0|\xv)<C_{II} P(H_1|\xv)$.

We need to note two things from here. First, the difference between a `probability',
which is the probability of an event given data; and a `likelihood', which is the
probability that we'd have the data we have given some event. This is nitpicking that
you don't really have to care about, except that here and there somebody will call you
on it. This may also clarify for you why some things are called one and some the
other.

Second, ?`remember the definition of a \ind{conditional probability}? It's based on the
statement $P(A\cap B)=P(A|B)P(B)$, and symmetrically, $P(A\cap B)=P(B|A)P(A)$.  Equating
the two and shunting $P(B)$ over gives us \ind{Bayes's rule}: $$P(A|B)={P(B|A) P(A)\over P(B)}$$
Applying Bayes's rule to the rejection test is easy: set $A=H_0$, $B=\xv$,
$P(A|B)=P(H_0|\xv)$, and $P(B|A)=L(\xv|H_0)$. Then:
\begin{eqnarray}
C_I P(H_0|\xv)&<&C_{II} P(H_1|\xv)\label{firstnp}\\
\nonumber\\
C_I {L(\xv|H_0)P(H_0)\over P(\xv)}&<& C_{II} {L(\xv|H_1)P(H_1)\over P(\xv)}\label{secondnp}\\
\nonumber\\
c&<& {L(\xv|H_1)\over L(\xv|H_0)}\label{thirdnp}
\end{eqnarray}

Inequality \ref{firstnp} is the rejection rule from above; Inequality
\ref{secondnp} uses Bayes's rule to insert the likelihood functions;
Inequality \ref{thirdnp} just does some cross-division, cancelling out the
$P(\xv)$s, and defining $c\equiv C_IP(H_1)/C_{II}P(H_0)$, i.e., everything
that doesn't depend on $\xv$. If you tell me the shape of the distribution
and some number $\alpha\in(0,1)$, then I can give you a value of $c$
such that Inequality \ref{thirdnp} is true with probability
$\alpha$. The test will then be: fail to reject $H_0$ iff Inequality
\ref{thirdnp} is true.

Neyman and Pearson\index{Neyman-Pearson lemma} proved\footnote{See
e.g. \cite[189--191]{amemiya:ez}, from whom I cribbed
this section.}  that this test is the `best' in the sense that for a
fixed type I error (fixed at $\alpha$), the probability of a type II
error is minimized. So we can design any test we like by just fixing $\alpha$
at a value we're comfortable with (custom says to use
90, 95 or 99\%), calculate a few likelihood functions, and we're assured
that we did the best we could regarding type II errors. You'll see that
many of the tests which follow will be of the likelihood ratio form,
and so type II errors pretty much never get mentioned, since they're
considered taken care of.




\subsection{The $\chi^2$ test \index{$\chi^2$ test}}

By itself, this is only useful for testing whether the sample variance
is equal to a given $\sigma^2$, because unless you assume $\sigma^2$ you
don't have it (unless you are doing a homework question that says something like `$\sigma^2$
is known'). Otherwise, in the case of multiple variables, you need to use
an $F$ test; see below.

$H_0$ asks, is ${\Qv'
\betav} = \cv$?  This is a very versatile question. E.g., $\Qv =
\vector{1\cr 0\cr 0}$ plus $\cv = [7]$ gives $H_0: \beta_1 = 7$. Or, $\Qv = \vector{1 \cr
-1}$ and $\cv = \vector{0}$ gives $H_0: \beta_1=\beta_2$. 
Or, say we want to test that $\beta_2 = 2\beta_3$. Then let $\Qv=\vector{0 \cr 1 \cr -2}$ and $\cv = 0$.
To test multiple hypotheses at once, simply stack the above:
\begin{equation}
\matrix{\Qv' = \vector{1 & 0 & 0  \cr
                0 &1 &-2}
                & \cv = \vector{7 \cr 0}}.\label{qc}\end{equation}

I dare
say that every linear hypothesis having to do with a mean or a $\beta$ can be
fit into this form.\footnote{If our concern is a mean, then set
$\Yv=$your data, $\Xv=$ a column of ones, and $\beta$ will then be
$\mu$.}
Define $q$ to be the number of constraints (columns
of $\Qv$), $N$ the sample size, and $K$ the number of parameters to be
estimated ($\betav$).

Now, if $H_0$ is true and we're using OLS, then $\Qv' \betav \sim {\cal
N}(\cv, \sigma^2 \Qv' (\Xv'\Xv)^{-1} \Qv)$,\footnote{For any other
method, the variance is $\Qv'$(the variance from section
\ref{cat})$\Qv$.} and we can use all of the above. The $\chi^2$
statistic now becomes:

\begin{equation}		\label{chi1}
{(\Qv'\hat\betav - \cv)' [\Qv' (\Xv'\Xv)^{-1} \Qv]^{-1} (\Qv' \hat\betav - \cv)
\over \sigma^2} \sim \chi^2_q
\end{equation}

If we're testing a variance, we'd have 

\begin{equation}		\label{chi2}
{{\bf u}' {\bf u} \over \sigma^2} \sim \chi^2_{N-K}
\end{equation}


\subsection{The $F$ test\index{F test}}

As above, we can divide equation \ref{chi1} by equation \ref{chi2} to give us a statistic
with an $F$ distribution

\begin{equation}	
{N-K\over q}
{(\Qv'\hat\betav - \cv)' [\Qv' (\Xv'\Xv)^{-1} \Qv]^{-1} (\Qv' \hat\betav - \cv)
\over \uv' \uv } \sim F_{q,N-K} \label{ftest}
\end{equation}	

If you have read this far, you know how to code all of the operations
in Equation \ref{ftest}.  But fortunately, 
Apophenia includes a function that will calculate Equation \ref{ftest}
for you.
\index{F test!apop\_F\_test@\cinline{apop\_F\_test}}
To do this, you will need to feed the function a $\betav$ vector, and a
matrix $\Qv'$ plus a vector $\cv$ that indicate the set of hypotheses
you wish to test. Notice that each {\em row} of the input matrix represents a
hypothesis. Also, recall that functions ending in \cinline{calloc} allocate
a matrix or vector and simultaneously set every element to zero, giving
you a blank slate to fill. Thus, if you already have a data set, and
would like to run an OLS regression and then test the joint hypothesis
of Equation \ref{qc}, here's what you'd type:
\begin{lstlisting}
apop_estimate *e = apop_OLS.estimate(data, NULL, NULL);
gsl_matrix *Q   = gsl_matrix_calloc(3,2);
gsl_vector *c   = gsl_vector_alloc(2);
gsl_matrix_set(Q, 0, 0, 1);
gsl_matrix_set(Q, 1, 1, 1);
gsl_matrix_set(Q, 1, 2, -2);
gsl_vector_set(Q, 0, 7);
apop_F_test(e, data, Q, c);
\end{lstlisting}

Included in the default output from a regression,
many stats packages run an $F$-test that reports the
joint hypothesis that all coefficients are different from zero. That is,
the $F$-test where $\Qv = {\bf I}$ and $\cv = {\bf 0}$. It is up to you
to decide whether this is a relevant statistic for the situation you are
dealing with, but because it is a custom to report it, some people expect it,
so
Apophenia facilitates this hypothesis test by assuming it is the default to assume when you to send in {\tt
NULL} variables: \cinline{apop\_F\_test(estimate, data, NULL, NULL)}. That is, if
$\Qv'=$\cinline{NULL}, Apophenia assumes ${\bf I}$ and if $\cv=$\cinline{NULL},
Apophenia assumes ${\bf 0}$.




\comment{
\subsection{The F-test}
But we have the whole variance-covariance matrix to work with, not just
the diagonal, so we can readily test any joint hypothesis that suits
our fancy. First, we need a means of expressing the hypothesis.
}

\subsection{The t-test} \label{ttest}
\index{t test!apop\_t\_test@\cinline{apop\_t\_test}}
\index{t test!apop\_paired\_t\_test@\cinline{apop\_paired\_t\_test}}
Among the most common and simplest questions one asks is: are two data
sets from the same process? Let \cinline{n\_side} be the income of households
drawn from the North side of town, North of the railroad tracks, and
\cinline{s\_side} be the income of households drawn from the South side of
town. Are North side incomes different from South side incomes? 

It so happens that the sum of two Normal distributions is also a normal
distribution, which means the the difference between two Normals is also
Normal, and with enough algebra, one can prove that a comparable difference of 
two t-distributions is also t-distributed. Notably, if $\mu$,
$\sigma^2$, and $n$ are the estimated mean, variance, and actual count
of elements of each data set,
$${\mu_a + \mu_b \over \sqrt{\sigma^2_a/n_a + \sigma^2_b/n_b}} \sim t_{n_a + n_b -2}.$$
Using the ingredients above, the reader could construct a function to
test the hypothesis that the number above, the $t$-statistic, is
different from zero for any given confidence level. Apophenia provides a
high-level function to do the work for you:

\begin{lstlisting}
apop_vector *a = gather_data("n_side");
apop_vector *b = gather_data("s_side");
apop_t_test(a, b);
\end{lstlisting}

Now let us say that the data is paired, in the sense that for each
element in the first set, there is an element in the second set that is
related; put another way, this means that for each element $a_i$, there
is a corresponding element $b_i$ such that the difference $a_i - b_i$
makes real-world sense. For example, we could look at student scores on
a test before a set of lessons and the same students after the lessons.
Then, rather than looking at the $t$-distribution for the before data
and comparing it to the $t$-distribution for the after data, we could
look at the vector of differences $a_i - b_i$ and test where zero falls
in the appropriate $t$-distribution. This is a more powerful test,
meaning that we are more likely to reject the null hypothesis of no
difference between the two vectors, therefore researchers prefer the
paired $t$-test over the unpaired version above. Apophenia provides the
\cinline{apop\_paired\_t\_test} function to run this test where appropriate.

\comment{
\section{Good ol' OLS}
This section would discuss how to write your very own \cinline{regress} function, which, as noted above,
just consists of solving for the $\beta$ in $(X'X)\beta = X'Y$. 
The last chapter showed us the code to find the betas with the smallest squared error; this section will
cover testing hypotheses about those betas.

\subsection{GLS} Generalized least squares refers to any method that uses a variance-covariance matrix
that isn't the identity matrix. Having written our \cinline{regress} function, it's almost trivial to
generalize to GLS. But the fun of GSL is in working out what that matrix should be.  This section would
give examples of favorites such as AR-1 processes from time series analysis (and hey, why not AR-$N$?).

The easiest thing to do is simply calculate the variances and covariances of the data itself:

%\begin{verbatim}
\begin{lstlisting}
#include "gsl_convenience_fns.h"
gsl_vector_view one_col, another_col;
gsl_matrix *var_covar=gsl_matrix_alloc(data->size2, data->size2);
int i,j;

for(i=0;i< data->size2; i++){
   for(j=0;j<= i; j++){
      one_col = gsl_matrix_column(data, i);
      another_col = gsl_matrix_column(data, j);
      covar = cov(&one_col.vector, &another_col.vector);
      gsl_matrix_set(var_covar, i, j, covar);
      if (j!=i) gsl_matrix_set(var_covar, j, i, covar);
   }
}
\end{lstlisting}
%\end{verbatim}

Now that we've got this variance-covariance matrix, $\Sigma$, we can
apply the formula $\beta_{GLS} = (X'\Sigma X)^{-1} (X'\Sigma Y)$.
}

\comment{
This gives us  $N\cdot (N+1)/2$ covariances that need to be calculated; your data is probably not up to
the task of providing enough information to allow this to be estimated, in which case you'll need to
compromise and impose some number of restrictions. 

For example, FGLS (Feasible GLS) assumes that the var-covar matrix is of the form 
$$\left[\matrix{
1 &\rho & \rho^2 & \rho^3\cr
\rho &1 & \rho  & \rho^2\cr
\rho^2 & \rho & 1 & \rho\cr
\rho^3 & \rho^2  & \rho &1\cr
}\right].$$
This makes life easy because we now only have one variable to estimate.
}

\section{Hypotheses about the variance}
Chi-squared tests and their utility. We need to calculate the sum of errors squared 
and then compare that to the Chi-squared table. 


\section{Testing the assumptions} Errors have to be normally distributed, or else
the whole OLS system here doesn't apply. Many stats package user's manuals
suggest plotting the errors and then squinting at the picture. Me, I'm a
fan of a slightly more scientific approach, which is based on the fact
that a Normal distribution has only two parameters: the mean and the
standard deviation. Everything else about the Normal distribution is
defined therefrom. Notably, the third moment is zero, and the fourth
moment is $3 \sigma^4$. We've already written everything we need to calculate all of these easily.

\comment{
\begin{verbatim}
#include "gsl_wrappers.h"

gsl_vector *errors = gsl_vector_alloc(beta_dot_x->size);
int i;
for (i=0;i< beta_dot_x->size; i++)
   gsl_vector_set(errors, i, gsl_vector_get(beta_dot_x, i) 
                                 - gsl_matrix_get(data,0,i))

double three_var_squared = 3* var(error)*var(error);
printf("The mean is: %g\n", mean(error));
printf("The var  is: %g\n", var(error));
printf("The skew is: %g\n", skew(error));
printf("The un-normalized kurtosis is: %g, and should be %g, for a difference of %g.\n", 
                         skew(error), three_var_squared,
                         skew(error)-three_var_squared);
\end{verbatim}
}

[I had some code here, but I don't like it. It just prints the first four moments of a data vector.]

You can either just eyeball this, and decide based on the scale of the variance
whether the skew and the kurtosis look like they're far off from where they should
be, or you could bootstrap the variance of the kurtosis, which would let you find
a confidence interval around $3 \sigma^4$ and state with a certain percentage
of certainty that the kurtosis is or is not where it should be. Hint: the CLT applies to the kurtosis,
so you know it has a Gaussian distribution.

\marginalia{Kurtosis minus three}{
There is some debate as to whether the kurtosis of a ${\cal N}(0,1)\equiv 3$
or $\equiv 0$. This is simply a definitional issue: some prefer to
normalize all reported kurtoses by subtracting three from the actual
fourth moment. For example, the GSL does the normalization; Apophenia does not.
Bear this in mind when interpreting kurtosis output. If you see a
negative kurtosis, then you are guaranteed that you are getting
normalized kurtosis, since a fourth power can never be negative.\index{kurtosis}}
