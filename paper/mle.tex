\chapter{Maximum likelihood estimation} \label{mle}
\index{maximum likelihood estimation|ff}
\index{MLE|see{maximum likelihood estimation}}

Maximum likelihood estimators (MLEs) are the bread and butter of
statistics. Most of the techniques we've handled so far are MLE
techniques, except mathematicians over the ages have found ways to hide
that fact from you. But if there isn't a nice, convenient way to get
around doing the maximization, you'll have to do it yourself. Fortunately,
the GSL has the {\tt gsl\_multimin} family of objects, to help you find
the optimal parameters.

\section{Why likelihood functions are great} There are two reasons, with four names.

\paragraph{MLEs achieve the Cramer-Rao lower bound} The CRLB of variance
is the inverse of the derivative of the derivative of the log of the
likelihood function. [I'll get around to writing it nicely later.] It's
called a `lower bound' because Mr.s Cramer and Rao proved that any
estimator of $\beta$ must have a variance greater than or equal to the
CRLB. Your favorite statistics textbook will also prove that for the MLE,
the variance is actually equal to the CRLB, meaning that we can not get
an estimator of $\beta$ which will have any smaller a variance.

\paragraph{The Neyman-Pearson lemma} There are two types of error we could
have with a hypothesis test: Type I is that we reject the null when it's
true; Type II is that we accept the null when it's false. The first type
is the one we focus on, because it's what we mean when we say that our
test has $\alpha=95\%$ confidence. What about Type II errors? Well, Neyman
and Pearson showed that a likelihood ratio test will have the minimum
possible Type II error of any test with the $\alpha$ that we selected. After establishing this fact, we
just ignore Type II errors.

\section{Writing your likelihood function} 
This section is about writing down the likelihood function for those situations where a linear least
squares function is not appropriate.
Writing down your function
is pretty darn straightforward, but there are details you'll need to take into account.\\
--Take logs. This is important from a practical standpoint because the product of a thousand probabilities
$\in (0,1)$ will quickly underflow your likelihood function. \\
--Follow the {\tt gsl\_multimin} template.\\

[I'll eventually write this out, but for now, here's an example:]


\begin{verbatim}
#include "gsl_convenience_fns.h"
double probit_likelihood(const gsl_vector *beta, void *d){
int		i;
long double	n, total_prob	= 0;
gsl_matrix *data 		= d;		//just type casting.

   dot(beta,data);    //a wrapper for gsl_blas_dgemm.
   for(i=0;i< data->size1; i++){
      n	=gsl_cdf_gaussian_P(gsl_vector_get(beta_dot_x,i),1);
      if (gsl_matrix_get(data, i, 0)==0) 	total_prob	+= log(n);
      else 					total_prob	+= log((1 - n));
   }
   return -total_prob;
}
\end{verbatim}

\section{Description: Maximum likelihood estimators} 
If you've written
down the function correctly in the last section, you'll have no problem
getting the GSL to find the optimal parameters given your data.  It helps
if you know the derivative of your data, which I'll also discuss a little
further.

\subsection{The GSL's multimin functions} The process of finding a minimum consists of trying a
value, picking a direction to move in, and checking whether the change is a good enough one. The GSL
gives you a function to do all of these things, which you'll have to put together to do a complete
minimization.

\section{Hypothesis testing: Likelihood ratio tests} Every test
in the last chapter was a likelihood ratio test---I just didn't tell
you what the likelihood function was. Those functions are easy because
they've been carefully studied and methods have been found to let you
calculate them without explicitly writing down a likelihood function
and calculating its value at various locations.

But if your data is at all interesting, then you'll need to write down
the likelihood function yourself.  Fortunately, we have a computer to
do the tedious math, unlike poor Mr. Gauss, who had no such conveniences.


\paragraph{What if you don't know the variance or distribution?} The main convenience of the canned methods of the last
chapter is that we have mathematical proofs telling us exactly what the distribution looks like. What if
our data doesn't fit the assumptions of any of the proofs in the textbook?

Then we bootstrap to find the distribution! I'll put an example here which puts together the techniques
from the last chapter and this one, constructing a likelihood function like the one above using the
bootstrap.

