\chapter{Databases} \label{sql}
\setsql %for code listings.

Structured Query Language and matrix operations are excellent complements. Some things, like
making a crosstab, are a bit of a pain using traditional techniques of
matrix manipulation. Meanwhile, some things just can not be done via SQL
queries. With both, your data analysis technique will be unstoppable.

As a very broad rule, try to do data manipulation, like pulling subsets
from the data or merging together multiple data tables, using SQL. Then,
as a last step, pull the perfectly formatted data into a GSL matrix and
do the statistical analysis. 

SQL is not nearly as complex a language as C, making it easy to learn
and one might even say outright pleasant to work with. Here is some valid SQL: {\tt
select age, gender, year from survey}. Why, that's almost proper English. It
goes downhill from there in terms of properness, but at its worst, it is
still not difficult to look at an SQL query and have some idea of what
the rows and columns of the output table will look like.

\section{Getting data into the database}
Apophenia provides a set of functions to facilitate querying databases.
It will open a single database, into which you can add all the tables you would like.

The first command you will need is \cind{apop\_open\_db}. If you give
it the name of a file, like \cinline{apop\_open\_db("study.db")}, then the
database will live on your hard drive. This is slower than memory, but
will exist after you stop and restart the program, and so other programs
will be able to use the file, you have more information for debugging,
and you can re-run the program without re-reading in the data.

Conversely, if you give a null argument: \cinline{apop\_open\_db(NULL)},
then the database is kept in memory, and will run faster but 
disappear when the program exits.


Unless your program is generating its own data, you will probably
first be importing data from a text file.  The \cinline{apop\_text\_to\_db}
function will do this for you; see the online reference for details. The
first line of the text file can be column names, and the remaining rows
are the data.


When you are done with all of your queries, run
\cinline{apop\_close\_db} to close the database. If you send the function a
one---\cinline{apop\_close\_db(1)}---then SQLite will take a minute to clean
up the database before exiting, leaving you with a smaller file on disk;
sending in a zero doesn't bother with this step. Of course, if your
database is in memory, it's all moot and you can forget to close the
database without consequence.

\section{Basic queries}
This section discusses some of the methods of data slicing and joining
that are easy using database queries, but difficult via
matrix operations.

For example, pulling a subset of your data is trivial in SQL. If
you need all of the data for those subjects over 175cm, you can ask for
it thusly:

%\begin{verbatim}
%#sql
\begin{lstlisting}
select * 
   from data 
   where height >=175;
\end{lstlisting}
%\end{verbatim}

You can read this like English (once you know that \sinline{*} means `all
columns'): it will find all of the rows in a table named \sinline{data}
where \sinline{height} in that row is greater than or equal to 175, and
return all the columns for those rows.


Generally, the \sinline{select} statment gives a list of columns that
the output table will have; the \sinline{from} clause declares where the
source data comes from; finally, there are modifiers or restrictions
(in this case \sinline{where}) on the rows to be output.  And that's
it. Every query you run will have these three parts in this order: column
specification, data source, row specification.\footnote{You may have no
row restrictions, in which case your query will just have the first two
parts and a null third part.} However, the means of specifying rows,
columns, and source data allows for a huge range of possibilities.

\paragraph{Select}
The \sqlind{select} clause will specify the columns of the table which will
be output. The easiest list is \sinline{*}, which means `all the columns'.

Other options:\begin{itemize}
\item Explicitly list the columns:\\ \sinline{select age, height, weight}
\item Explicitly mention the table you are pulling from:\\
\sinline{select data.age, data.height, data.weight}\\
This is unnecessary here, but will become
essential when dealing with multiple tables below.
\item Rename the output columns:\\
\sinline{select data.age as age, height as height\_in\_cm}\\ 
Notice that if you do not rename \sinline{data.age as age}, then you will need to use \sinline{data\textbackslash.age} in future
queries, which is a
bit annoying.
\item Generate your own new columns:\\
\sinline{select height, weight, (height/weight) as hwr}\\
The \sinline{as hwr} subclause is again more-or-less
essential if you hope to easily refer to this column in the future. 
\end{itemize}

\paragraph{From} The \sqlind{from} clause specifies the tables from which
you will be pulling data. The simplest case is a single table: \sinline{from data\_tab}, but you can specify as many tables as necessary: \sinline{from data\_tab1, data\_tab2}. 

You can alias the tables, for easier reference. The clause 
\sinline{from data\_tab1 d1, data\_tab2 d2} gives short names to both tables, which can
be used for lines like \sinline{select d1.age, d2.height}. 

Aliasing is generally optional but convenient, but one case where
it is necessary is when you are joining a table to itself.
For now, simply note the syntax: \sinline{from data t1, data t2} will let
you refer to the \sinline{data} table as if it were two entirely independent
tables. 

Notice, by the way, that when we aliased something in the \sinline{select}
section, the form was \sinline{select long\_col\_description as lcd}, while
in the \sinline{from} section there is no \sinline{as}: \sinline{from long\_file\_name lfn}.

\paragraph{Where}
The \sqlind{where} clause is your chance to pick out only those rows which
interest you. With no \sinline{where} clause, the query will return one line
for every line in your original table (and the columns returned will
match those you specified in the \sinline{select} clause).

Join as many subclauses as you wish, using the Boolean operators you
know and love: \sinline{where ((age > 13) or (height >= 175)) and (weight == 70)}.


\paragraph{Pruning rows with \sinlinetwo{distinct}} The \sqlind{distinct}
keyword will tell the SQL engine that if several rows would be exact
duplicates, to return only one copy of that row. Place it after 
\sinline{select}: 
\begin{lstlisting}
select distinct age, height, zip
from table
where age > 20
\end{lstlisting}
The  \sinline{distinct} word prunes the rows, but is placed in the 
\sinline{select} portion of the program.  This reads more like English, but it
breaks the story above that the \sinline{select} statement specifies the
columns and the \sinline{where} statement specifies the rows.

\subsection{Joining}
If you specify two tables in your \sinline{from} line, then, lacking any
restrictions, the database will return one joined line for every pair of lines.
Let table 1 have one row with data \{1, 2, 3\} and table 2 have one row
with data \{a, b, c\}; then \sinline{select * from table1, table2} will
produce an output table with every combination:
\sinline{\\
1 a\\
1 b\\
1 c\\
2 a\\
2 b\\
2 c\\
3 a\\
3 b\\
3 c}. 

If \sinline{data1} has 15 lines, and \sinline{data2} has 8 lines, then
\sinline{select * from data1, data2} will return 15$\times$8 = 120 lines.
Such a product quickly gets overwhelming: joining a thousand-row table
with another thousand-row table will produce a million rows.

Thus, the \sinline{where} clause becomes essential. Its most typical use is
when one column in each table represents identical information. For
example, say that one data source gave you mean income by ZIP code,
while another gave you mean heights by ZIP code. Joining the two tables
would give you a multitude of lines which include height from a ZIP code
in Alaska and an income for a ZIP in Kansas; clearly, the only ones you
are interested in are those where the ZIP code is the same in both data
sets. Here is a query that would keep only those lines which make
sense:

\begin{lstlisting}
select t1.income, t2.height
   from econ_data t1, health_data t2
   where t1.zip == t2.zip_code
\end{lstlisting}

You can see that using the table-dot-column format for the columns is now
essential. Also, notice that we did not have to include either ZIP code
in the output if it is not used later. If you do want to include a ZIP
code, then you can use either \sinline{t1.zip} or \sinline{t2.zip\_code}; since
the two will be by definition identical, using both will be a waste of
space (but may help to give you a sense of security that things are working).

\paragraph{Speeding it up}
You can ask the SQL engine to create an index for a table that you
intend to use in a join later. The command:\sindex{indexes}
%\begin{verbatim}
\begin{lstlisting}
create index econ_index on econ_data(zip)
create index econ_index2 on econ_data(age)
\end{lstlisting}
%\end{verbatim}
would index the table \sinline{econ\_data} on the ZIP code and age columns.
The name of the index, \sinline{econ\_index}, is basically irrelevant and can
be any gibberish which sounds nice to you. Once you have created this
index, a join using any of these three columns goes {\sl much} faster.
For the query above, you will want to create an index for both {\tt
econ\_data(zip)} and \sinline{health\_data(zip\_code)}.

There is standard SQL syntax for indexing multiple columns, 
(\sinline{create index e\-con\_\-in\-dex on econ\_data(zip, age)}), but as a
practical matter, SQLite really only cares about the first item in the
list, so you are better off creating several indices as above.

\subsection{Aggregation}
Here is how to get the number of rows in your data table:
%\begin{verbatim}
\begin{lstlisting}
select count(*) as row_ct 
from data;
\end{lstlisting}
%\end{verbatim}
This produces a table with one column and one row, listing the total number of rows in the \sinline{data} table.

\marginalia{Crosstabs}{In the spreadsheet world, we often get tables
where the X-dimension is the year, the Y-dimension is the location,
and the $(x,y)$ point is a measurement taken that year at that location.

Conversely, the most convenient form for this data in a database is
three columns: year, location, statistic. [How would you write a query
such as \sinline{select statistic from tab where year < 1990} if there
were a separate column for each year?] 

Converting between the two is an annoyance, and so Apophenia provides
functions to do conversions back and forth:

\cindex{apop\_db\_to\_cross\-tab} \cindex{apop\_cross\-tab\_to\_db}
\cinline{apop\_data *crosstab = apop\_db\_to\_cross\-tab("table\_name", "year", "location", "statistic");}\\
and
\cinline{apop\_crosstab\_to\_db(crosstab, "new\_table\_name");}
}

You probably want more refinement than that; if you would like to
know how much data you have in each ZIP code, then use the \sqlind{group
by} clause to say so:
%\begin{verbatim}
\begin{lstlisting}
select zip, count(*) as ct_per_zip
from data
group by zip;
\end{lstlisting}
%\end{verbatim}
This will group the rows into groups, where each group has the same ZIP
code, and return a single line for each group.


The two other aggregation commands which will be useful to you are {\tt
sum()} and \sinline{average()}. These take an existing row as an argument.
For example, to search for height disparities by ZIP code, try:
%\begin{verbatim}
\begin{lstlisting}
select  average(height) as avg_height
from data
group by zip;
\end{lstlisting}
%\end{verbatim}
Notice that we don't have to include ZIP code as a column in the output
if we are not interested in it.

Feel free to specify multiple {\sl group by} clauses; this creates what
some call a \ind{cross tab}. For example, to get average income by
height and ZIP code: 
%\begin{verbatim}
\begin{lstlisting}
select  zip, height, average(income) as avg_income
from data
group by zip, height;
\end{lstlisting}
%\end{verbatim}

When you want to analyze the crosstab, you will be very
interested in the \ttind{a\-pop\_\-db\_\-to\_\-cross\-tab} function, {\em qv}.

The function \sinline{sum()} works the same. That's all the aggregators you
get in standard SQL, but you can specify more than just a column name, so you can get
away with a few more. Here is the variance of income in each subgroup:
%\begin{verbatim}
\begin{lstlisting}
select  sum((income - average(income))*(income - average(income))) 
               as income_variance
from data
group by zip, height;
\end{lstlisting}
%\end{verbatim}

\paragraph{Apophenia's SQL extensions}
Apophenia saves you from having to do calculations like the above by
defining the sort of aggregations that statisticians use, such as
moments. You can use the following aggregation functions in an SQL
call that you make from Apophenia: 
\sinline{stddev var variance skew kurt kurtosis}. There are redundancies so that you don't have to remember
whether to abbreviate or not.

This may also be a good time to mention three more extentions to
standard SQL: \sinline{sqrt, pow, exp, log}. These functions do what you
expect: \sinline{select log(income), pow(in\-come, 3), sqrt(income) from data} will give
you a table with the log of incomes and, for what it's worth, incomes
cubed and the square root of incomes.

Remember: these functions are Apohenia extensions, so be careful not to
use these added functions if you ever hope to use your SQL queries in
another context. If you want to stay standard, you call your data into
a table and use \ttind{apop\_vector\_log}, \ttind{apop\_vector\_exp},
\ttind{apop\_mean}, \ttind{apop\_var}, et cetera to get the desired
statistics on the matrix side.

\subsection{Stacking tables}
You can think of joining two tables as setting one table to the right of
another table. But now and then, you need to stack one on top of
the other. There are four keywords to do this.

\paragraph{\cinlinetwo{Union}} For example, \ttindex{Union}
\begin{lstlisting}
select id, age, zip
from data_set_1
UNION
select id, age, zip
from data_set_2
\end{lstlisting}
will produce the results of the first query stacked directly on top of
the second query. Feel free to use subqueries or what have you, but be
careful that both tables have exactly the same number of columns.

\paragraph{\cinlinetwo{Union all}}\ttindex{Union all} If a line is duplicated in both tables,
then the \cinlinetwo{union} operation throws out
one copy of the duplicate lines, much like \sinline{select distinct}
includes only one of the duplicates. Replacing \sinline{UNION} with \sinline{Union all} will retain the duplicates.

\paragraph{\cinlinetwo{Intersect}}\ttindex{Intersect} As you can guess,
putting \sinline{intersect} between two \sinline{select} statements
returns a single copy of only those lines that appear in both tables.

\paragraph{\cinlinetwo{Except}}\ttindex{Except} This is the subtraction: it returns only
elements from the first table that do not appear in the second. Notice
the asymmetry: absolutely nothing in the second table will appear. 

\subsection{Creating tables} There are two ways to create a table. One
is via  a \ttind{create} statement and then an \ttind{insert} statement
for every single line. 

\begin{lstlisting}
begin;
create table newtab(name, age);
insert into newtab("Joe", 12);
insert into newtab("Jill", 14);
insert into newtab("Bob", 14);
commit;
\end{lstlisting}

The begin-commit wrapper, by the way, means that everything will happen
in memory until the final commit. The program may run faster, but if
the program crashes in the middle, then you will have lost everything.
The optimal speed/security trade-off is left as an exercise for the
reader.

If you have hundreds or thousands of \sinline{insert}s, you are almost certainly better off putting the data
in a text file and using \ttind{apop\_text\_to\_db}. The form
above is mostly useful in situations where you are creating the table in mid-program; see the example on page \pageref{createeg}.


The other method of creating a table is by saving the results of a
query. Simply append \sinline{create table newtab as} at the head of the
query you would like to save: 
%#sql
\begin{lstlisting}
create table tall_people as
select name, age, height 
from people
where height > 175
\end{lstlisting}



\subsection{More trickery} Frankly, SQL is not very good for
creating new data from the existing. But the few things that you can do
are very easy.

\paragraph{A time lag} Here is a simple example:
\begin{lstlisting}
select year, (t1.income - t2.income) as diff
   from data t1, data t2
   where t1.year=(t2.year -1);
\end{lstlisting}


The \sinline{where} clause will line up the table with a copy of itself
lagged by one year, and then the \sinline{diff} variable will be the
one-year change in income.


\paragraph{Subqueries}

One of SQL's nicest tricks is that it allows for the input tables to be
queries themselves. For example: what is the mean number of observations per
ZIP code?

You first need to get a \sinline{count(*)} grouped by ZIP, and then get an
average of that. You could run a query to produce a table of counts,
save the table, and then run a query on that table to find the averages.
Here is how to do all of this in one query: 
\begin{lstlisting}
select average(ct) 
from (select count(*) as ct
      from data
      group by zip)
\end{lstlisting}
The query inside the \sinline{from} clause will return a table, and even
though that table has no name, it can be used as a data source like any other
table. So the example will first produce a table with one line for each
ZIP code, listing the number of observations it has, and then it will
take that table and find the average of its \sinline{ct} column.

Everywhere you would put a table name, you could put a query 
instead. Of course, the query output has no name, so you may need to
alias the result: \sinline{from (select ...) t1} will allow you to refer to
the query's output as \sinline{t1} elsewhere in the query.

\subsection{Folding queries into C code} 


\paragraph{apop\_query} Now for the queries. The simplest function
is \cind{apop\_query}, which takes a single text argument: the
query. For example,
\setc
\begin{lstlisting}
apop_query(" select *        \
               from data     \
                  where height >=175;");
\end{lstlisting}

A string is easiest for you as a human to read if it is
broken up over several lines; to do this, end every line with a
backslash, until you reach the end of the string. The line above runs
the query and returns nothing. [Not very useful for a \sinline{select}, but perfect for
\sinline{create} or \sinline{insert}.]

\cinline{apop\_query} accepts \cinline{printf}-style arguments. For example,
say that you calculate a minimum height earlier in the program, and want
to write a query that uses that variable:
\begin{lstlisting}
int min_height	= 175;
apop_query(" select *     \
               from data     \
               where height >=%i;", min_height);
\end{lstlisting}

\paragraph{apop\_query\_to\_data} \cindex{apop\_query\_to\_data}
This function takes both a gsl\_matrix and a query string. It will run
the query and return the resulting table for your analysis.  
\begin{lstlisting}
apop_data *tall_people;
int min_height	= 175;

tall_people = apop_query_to_data(" select *     \
                                   from data     \
                                   where height >=%i;", min_height);
\end{lstlisting}
will query \sinline{data} for its information, allocate \sinline{tall\_people} to be an
appropriately-sized matrix, and fill it with the data.

There are \cinline{apop\_query} functions for any type you could desire,
including \cind{a\-pop\_\-query\_\-to\_\-ma\-trix} to pull a query to a {\tt
gsl\_matrix}, \cind{apop\_query\_to\_chars} to pull a query into
a two-by-two table of strings, \cind{apop\_query\_to\_vector}
and \cind{apop\_query\_to\_float} to pull the first column or first
number of the returned table into \cinline{gsl\_vector} or a \cinline{float}.

\paragraph{Data to db} For the other direction, there are the plain old
print functions: \cind{a\-pop\_\-da\-ta\_\-print} and
\cind{apop\_matrix\_print}. If the global variable
\cinline{apop\_opts.output\_type == 'd'}, then these functions will dump the
given data to the given file:
\index{apop\_opts!output\_type@\cinline{output\_type}}
\begin{lstlisting}
apop_opts.output_type == 'd';
apop_data_print(datatab, "my_data");
\end{lstlisting}
Say that tomorrow you decide you would prefer to have the data dumped to
a file; then just change the \cinline{'d'} to an \cinline{'f'} and away you go.

There is one inconvenience to this convenience, however: dots are popular
in file names, e.g. \binline{dump.csv}, but are not valid in table names. If
you try to use that file name as a table, SQLite will think you are
referring to a database named \sinline{dump} with a table named \sinline{csv},
and will will give you an error in the way of \binline{no database named ``dump''}.

\paragraph{Merging databases} Apophenia handles exactly one database at
a time. However, there are times when you will want to use data
from separate databases, in which case \cind{apop\_\-db\_-merge} and
\cind{apop\_\-db\_-merge\_\-table} will help.  The first command
operates on every table in the database, while the second operates only
on one table that you specify; see the online reference for details.

If the table exists in the new database but not in the currently open
one, then it is simply copied over. If there is a table with the same
name in the currently open database, then the data from the new table is
inserted into the main database's table with the same name. [The function
just calls \sinline{insert into main.tab select * from merge\_me.tab}.]

In-memory databases are faster, but at the close of the program, you
may want the database on the hard drive. To get the best of both
worlds, use an in-memory database for the program, and then as the last
line of the program, merge the database with an on-disk database:
\begin{lstlisting}
int main(void){
    apop_db_open(NULL); //open a db in memory.
    do_hard_math(...);
    remove("on_disk.db"); //standard C command to delete a file.
    apop_db_merge("on_disk.db");
}
\end{lstlisting}
By removing the file before merging, we prevented the duplication of
data (because duplicate tables are appended to, not overwritten).

\section{Some examples} 
Here are a few examples of how C code and SQL calls can neatly interact.

\subsection{Taking simulation notes}\label{createeg}
Say that you are running a simulation and would like to take notes on
its state each period in a neat format. The following code will open a
file on the hard drive, create a table, and add an entry each period.
Notice that there is no begin-commit wrapper, so if you get tired of
waiting, you can halt the program and walk away with
your data.\footnote{Sometimes such behavior will leave the database in
an erratic state. If so, try the SQLite command \sqlind{vacuum}.}

%#c
\begin{lstlisting}
int     i;
double  sim_output;
apop_db_open("sim.db");
apop_table_exists("results", 1);    //See below.
apop_query("create table results (period, output);");
for (i=0; i< max_periods; i++){
    sim_output = run_sim(i);
    apop_query("insert into results(%i, %g);", i, sim_output);
    }
apop_db_close(0);
\end{lstlisting}

The \cind{apop\_table\_exists} command checks for whether a table
already exists. If the second argument is one, as in the example above,
then the table is deleted so that it can be created anew subsequently;
if the second argument is zero, then the function simply returns the
answer to the question ``does the table exist?'' but leaves the table
intact if it is there. This is the dignified way to delete a table from
a database, because it will not return an error if the table you are
trying to delete does not exist.

\subsection{Dummy variables}
Here is a neat trick: using SQL's \cind{case} a few dozen times, we can turn a
variable which is discrete but not ordered (such as district numbers in the
following example) into a series of dummy variables. It requires writing down a
separate \sinline{case} statement for each value the variable could take, but that's
what \cinline{for} loops are for.

\codefig{dummy}{A sample of using SQL to create dummy variables}

Figure \ref{dummy} uses the \sinline{case} keyword to create a series of
columns that are either one or zero---dummy variables.
We first query out the list of districts, then we write a
select statement with a line \sinline{case district when district\_no then 1 else 0} for
each and every district\_no. You can then run your regression on the output of the
query without any further manipulation.

Notice that the for loop goes from i=1, not i=0; this is because when including
dummy variables, you always have to exclude one value, which will be the baseline;
using i=1 means district[0] will be the baseline.

Now that I've shown you that neat trick, don't do it; instead use
\cind{apop\_\-pro\-duce\_\-dum\-mies} on the matrix side.
%\cindex{apop\_produce\_dummies}

\subsection{Easy t-tests}
Say we have a set of observations of our sample's years of education, and their annual income. We want to
know if getting that grad school education is {\it really} worth it. The null hypothesis is: (Income for
people with less than 16 years of education) $\leq$ (income for people with greater than or equal to 16 years
of education).

That first data set is:
\begin{lstlisting}
gsl_vector	*undereducated;
   undereducated = apop_query_to_vector(
      "select income from survey \
      where education < 16");
\end{lstlisting}
while the second group is:
\begin{lstlisting}
gsl_vector	*overeducated;
   overeducated = apop_query_to_vector(
      "select income from survey \
      where education >=16");
\end{lstlisting}

Here is a factoid for you: incomes are usually distributed log-normally, so we should do a t-test on the
log of income. We'll do one with Apohenia and one with the raw GSL:\cindex{apop\_vector\_log}
\begin{lstlisting}
apop_vector_log(undereducated);

int i;
for(i=0; i< overeducated->size; i++)
   gsl_vector_set(overeducated, i, 
               gsl_sf_log(gsl_vector_get(overeducated, i)));
\end{lstlisting}

The GSL 
provides functions to find the mean and variance of a vector, and 
Apophenia provides some convenience wrappers for those functions:

\begin{lstlisting}
double	mean_over, mean_under, var_over, var_under;
mean_over  = gsl_stats_mean(overeducated->data, overeducated->stride, overeducated->size);
var_over  = gsl_stats_variance(overeducated->data, overeducated->stride, overeducated->size);
mean_under = apop_vector_mean(undereducated);
var_under  = apop_vector_variance(undereducated);
\end{lstlisting}

The other factoid you'll need is that the difference of two normal
distributions is also normal, and the variance of the difference is
the sum of the two original variances.\footnote{Your stats textbook
will tell you that the sum of two Normals is normal: ${\cal N}(\mu_1,
\sigma_1) + {\cal N}(\mu_2, \sigma_2) \sim {\cal N}(\mu_1 + \mu_2,
\sigma_1 + \sigma_2)$. Subtracting a ${\cal N}(\mu_2, \sigma_2)$
is exactly equivalent to adding a ${\cal N}(-\mu_2, \sigma_2)$, so we
get the result in the text.} That is, assuming the counts are large
enough that the Normal approximation is OK, we can write down:

%\begin{verbatim}
\begin{lstlisting}
double	test_me = gsl_cdf_gaussian_P(mean_over-mean_under, 
                                       var_over+var_under);
\end{lstlisting}
%\end{verbatim}

and \cinline{test\_me} is the probability that the difference between the
means is less than or equal to zero. If \cinline{test\_me} turns out to be greater
than your preferred confidence interval, (e.g., 95\%), then reject the
null and go to grad school. Else, there isn't enough information to
distinguish between the two with confidence.

\subsection{No, easier} The above example gave you line-by-line
control, but say the boss gave you ten minutes to get out a
t-test. Here is a complete program:
\index{t test!apop\_t\_test@\cinline{apop\_t\_test}}

\begin{lstlisting}
#include <apophenia/headers.h>

int main(void){
    apop_db_open("education_data.db");

    gsl_vector	*undereducated = apop_query_to_vector(
      "select log(income) from survey \
      where education < 16 and income>0");

    gsl_vector	*overeducated = apop_query_to_vector(
      "select log(income) from survey \
      where education >=16 and income>0");

    apop_t_test(undereducated, overeducated);
    return 0;
}
\end{lstlisting}

This is stats-package level work: open the database, query out two
columns, and call a $t$-test function that does things
we don't have time to think about. Those who report incomes less than or
equal to zero, we just throw out.\footnote{You may be able to do better
by querying out incomes, and then calling
    \cinline{apop\_vector\_log(overeducated)} and {\tt
    apop\_vector\_log(undereducated)} to get the logs. Then you will have
    \cinline{GSL\_NEGINF}s where you had zeros, which means you will at least
    know exactly where your problem is. Of course, the $t$-test will
    still fail because of the infinite variances.}

See page \pageref{ttest} for more on the $t$-test and this function.

\setc %for code listings.
