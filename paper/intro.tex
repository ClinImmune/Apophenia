\chapter[Introduction]{Statistics in the modern day}
\iftwocol\begin{multicols}{2}\fi


\comment{In case it is not obvious to you, we can not do statistical analysis without
computers.}
I'll start with the bad news:
being a statistician in the modern day means being a programmer. 
Many find this disheartening---we'd all rather be under a tree with a
book than staring at a computer screen all day.
But the mathematical explanation of a statistical procedure is
really just pseudo-code, which we can make operational by translating
it into real computer code. 

I wrote this book to help you make that translation. The first focus
is purely mathematical: we need to select techniques bearing in mind
that they will eventually be code. The second focus is about practical
coding: your life is short, and you want to spend as little of it as
possible trying to coax systems into complying with your wishes and 
learning yet more computer languages.

The good news is that
doing math with a computer is unfettering. Instead of using restrictive
regression techniques designed before computers, we can use techniques
built around computing thousand-term likelihood functions, or taking
millions of random samples. These techniques originally appeared in
the textbooks as just theory, then eventually with a caveat that these
techniques are possible but computationally intensive. Now computations
are cheap, and we can use these techniques as we would their simplified
brethren.

In my own pain-filled experience, the best way to operationalize
statistical concepts is using C; the GNU Scientific Library,  which
facilitates the linear algebra and the Gaussian distribution work;
and SQLite, a package which facilitates handling large data sets. This
book will cover the basics of these components and the manner in which
we can translate from the mathematical language to the language these
libraries speak.

The book is a complement to the Apophenia library, a set of functions 
based on the GSL and SQLite, intended to simplify the hard parts of
using these packages for statistical analysis. You could do all of the
analyses in this book without Apophenia, but you would find yourself
rewriting many of its functions. 
\comment{
Most stats packages include a manual that attempts to instruct the reader,
and an alphabetical-order reference providing detailed usage notes for
functions and structures.  You are reading Apophenia's manual now,
and the reference is online at \url{http://apophenia.info/doc}.}

This book falls in a mid-range between low-level numerical
computation and high-level use of prepackaged functions. For example,
{\sl Numerical Recipes in C} \citep{recipesinc} is a classic text
describing the algorithms for seeking optima and efficiently calculating
determinants. Being such a classic, there are many packages such as the
GSL that implement the algorithms from {\sl Numerical Recipes}, and this
book will build upon rather than replicate their effort. At the other
end, an abundance of texts will explain to readers how to do basic
statistics using the stats package of the month; this book lists the
one-line commands at this level, but it also breaks open the black boxes
and shows how those computations are done, so the reader can modify them
according to the vagaries of real-world data.

\paragraph{The intended audience}
I assume that you have had a statistics class or two already, and that
you have an encyclop\ae{}dic stats textbooks handy, so this
book will provide some utility as a review, a reference, and perhaps a
different perspective,
but there are few proofs and no pretentions to comprehensive coverage of
classical probability and statistics.

I also assume that you are computer-literate, and have experience writing
scripts in a stats package. The chapter on C (Chapter \ref{c_crash})
focuses on the techniques and features that are central to C but are
missing or underused in script-oriented languages. Those who are entirely
new to the idea of writing a sequence of instructions for a computer
may want to warm up with a few online tutorials.

If you are well-versed in C and SQL (Structured Query Language), then
the later parts of Part I and all of Part II will be immediately useful.

If you are not well-versed in C and SQL, then expect to take some 
time getting comfortable with these languages before doing statistics
with the techniques here becomes easy.

That is, there is an up-front investment to the methods here, and a
long-term payoff. If you expect that you will never do anything more
difficult than a linear regression on well-formed \ind{iid} data, or
if you are trying to slog through your department's stats requirement
so you can never look at another data set again, then by all means, put
this book down and get a copy of the easiest stats package you can get
away with. But if you expect to be doing statistical work for a larger
part of your career, such that problems of extensibility, portability,
and scaling may be looming in your future, then this book is for you.

It is not a short ride, but when you are done with all this---and this
is no exaggeration---you will have the tools to implement any technique
in classical statistics in existence today, on any data set, no matter
how large or exceptional.

\section{The programming covered}
\comment{
Using C is a mix of low-level and high-level
work. You will be allocating memory, telling the computer exactly where
to shunt its electrons. But, thanks to the efforts of tens of thousands
of programmers before you, you will have the benefit of functions that
will find the minimum of a function or calculate characteristics of a
Normal distribution, without having to remember Newton's method or
the equation for the Gaussian distribution.}

Part I is an overview of techniques for scientific computing: how to get
your computer to efficiently shunt large data sets and invert matrices.

It starts in Chapter \ref{c_crash} with a tutorial on the C programming
language.  The chapter covers nothing more difficult than than adding
columns of numbers, but it provides the basic grammar and structure that
the remainder of the book will use to do statistical analysis. Notably,
it introduces the idea of functions, and how one can build functions or
use existing functions to execute increasingly advanced routines.

\paragraph{Dealing with large data sets} Matrix-oriented packages,
(including the GSL) are not well-suited for data sets from agencies like
the US Census, that can be literally gigabytes of data.
C may seem unfriendly, but
those languages designed around dealing with large data sets tend
to be even more draconian---for example, SAS's data input command
is {\tt card}, referring to the punch card it expects you to put in the
hopper.\footnote{SAS, by the way, gets its facility with large data sets
by providing a front-end to SQL queries. You can enter SQL directly into
SAS if you so choose.}

Meanwhile, database systems are designed from the ground up to do nothing
but let you efficiently retrieve what you need from huge amounts of data.
But being so purpose-specific, there is no way to do statistics in
a database. Fortunately,
we are working in C, so we can have the best of all worlds. The techniques
covered in this book (and the method that Apophenia facilitates) is to
read all of your data into a database, and then query what you need to a
matrix, as you need it. This requires learning a new syntax, \ind{SQL},
which is decidedly neither Beautiful nor Perfect, and adds a level of
complication on top of what you are already dealing with. But the ease
of manipulating data sets offered by SQL is very much worth its short
learning curve. Chapter \ref{sql} presents a tutorial of SQL syntax to
get you up to speed. 

Apophenia uses the \ind{SQLite} library, which will
give you all the database functionality you need. Those of you who 
have data which is already handled
by another database engine will easily be able to adapt the techniques
used here, but will need to brush up on the details of how to access
your site's database. Since you are using C instead of a stats package,
you are guaranteed that there is an interface out there that you can
download and incorporate into your programs.

\paragraph{The computation engine} The \ind{GNU Scientific Library}
includes tools for all of the procedures commonly used in statistics,
such as linear algebra operations, looking up the value of F-, t-, $\chi^2$-
distributions, and finding maximia of likelihood functions. These
will be the building blocks for analysis. Chapter \ref{linear_algebra}
presents some basics for using the GSL, such as matrix manipulation and
linear algebra.

\paragraph{Apophenia}
Statistics textbooks could be much more brief. They could open
with notes on linear algebra and probability, and then conclude ``The
derivation of statistics from these basics is left as an exercise for
the reader.'' Similarly, the GSL includes all that we need to do
statistics, but in a raw form that is still a long way from the
sort of data analysis and hypothesis testing we want to do. Apophenia,
presented in Chapter \ref{apop}, is a library of structures and functions
that binds together the GSL and SQLite to facilitate data handling
and statistics.

\paragraph{Pretty pictures} One thing C is not really good for is drawing
pretty pictures. This is not to be belittled, since those pictures
can be very persuasive. Consistent with the rest of this book, plotting
is done via \ind{Gnuplot}, a program which is freely available for
the computer you are using right now. Gnuplot is highly automatable, so once
you have a plot you like, you can have your C programs autogenerate
it or manipulate it in amusing ways, or can send your program to your
colleague in Madras and he will have no problem reproducing and modifying
your graphs. Chapter \ref{gnuplot} will cover creative uses of
Gnuplot to display data.


\input why_c.tex



\section{The stats covered} 
\comment{
As powerful as we like to think modern statistics is, it has a 
limited set of tricks at its disposal. For the purposes of this book,
I will divide them into three broad categories, which are broad enough
to cover the great majority of classical statistics (in fact, they overlap).

{\it Projections.} This includes the overused OLS linear regression (which is
a projection of your data onto a line), and the underused factor analysis
techniques. These are the techniques we humans use to reduce too many
dimensions down to something we can comprehend and even draw a picture of.

{\it Gaussian distribution tricks.} The Central Limit Theorem says that
a very wide range of statistics will have a Normal distribution. Square
the Normal, and we have a Chi-squared distribution. Take the ratio of two
Chi-squared distributions, and we have an F distribution.  The sum of
a random number of Normally distributed variables will have a Laplace
distribution.

{\it Likelihood function tricks.} Once we know the probability of an
event, probably thanks to a Gaussian distribution trick, we can then
write down the likelihood that a given set of parameters would bring
about the data we see, and then find the parameters that maximize this
likelihood. MLEs have the pleasing property of meeting the Cramer-Rao
lower bound, and the ratio of likelihood functions has the pleasant
property embodied in the Neyman-Pearson lemma, making them the basis of
hypotheses testing.

\subsection{The goals of statistics}
Let's settle an
important fact about statistical analysis: it proves nothing, only
persuades. A large part of this is that we are looking for causal stories
about the world, but there is no statistical technique (and never will be)
that proves \ind{causality}. Further, there is always a way to rewrite a model
or re-draw the data so that a rejected hypothesis is not rejected, or vice versa.
But some results are more robust to tweaking 
than others, and some models are just plain more persuasive than others.

Within the overall goal of persuasion, we}
We can subdivide the goals of statistical analysis into two parts. The
first is to just say something interesting about a data set. This is often
model-free; for example, you may just want to show that two variables
are highly correlated, or that the data can mostly be summarized by three
dimensions.

The second part of statistical analysis is hypothesis testing, in which
we calculate a parameter of the data and then make a claim about that
parameter.  Having observed in the last paragraph that the correlation
coefficient of two variables is large, perhaps we would like to show
that that correlation coefficient is almost certainly different from
zero. More often, we have parameters in a model that we have written,
and would like to make claims about those parameters.

The techniques used for the two goals above are entirely different. For
example, there are any of a number of distributions listed in the
average stats textbook.  Poisson or Binomial distributions are used
only for describing data culled from the real world, while the Chi-squared
distribution or the F distribution are used only for testing hypotheses
about parameters we have written down ourselves (such as the correlation
coefficient, which has an F distribution). \citet[p 141]{kmenta} points out
that no notable natural populations have a Chi-squared distribution.
\index{chi squared distribution@$\chi^2$ distribution}

Part II of this book will open with a chapter on describing data,
Chapter \ref{projections}, using simple calculations or projection techniques. Then, Chapter
\ref{gauss} covers the hypothesis-testing half of the story, including 
methods of comparing your data to various Gaussian
distributions, such as the $t$ test, F test, and $\chi^2$ test, 
linear regressions, and goodness-of-fit tests.


\paragraph{General theorems and their special cases}\index{Ordinary Least
Squares|(}
Here is another way to divide the theorems that underly
statistics into two classes. The first class includes general
results about likelihood functions, that 
apply almost universally, but require a huge
amount of computational power to arrive at a result. The second class
consists of special cases of these general results, such as the theorems
underlying OLS regressions; these results impose more
assumptions, but are much easier to calculate, so that the results could
be used a century before computers were invented.\footnote{Historically,
the special cases came first and then were shown to be special cases of
more general results. Nonetheless, the rate of adoption of the general
cases has been in step with the speed of computing equipment.}

We spend most of our time in school learning the special cases,
because a few decades ago, this was the only class of results which
civilians had the computing power to use. This is when the stats packages
that are so prevalent today came to the fore, automating the tedium of
applying these special case results. The technology had an immediate
influence on how people did research: they applied those darn special
cases to everything, and one would have a hard time finding an issue of an
academic journal today that doesn't have at least one OLS regression.
Everything in the world has become a linear process.

This book is about using OLS only when it is applicable. 
The Central Limit Theorem tells us that yes, errors often are
Normally distributed; and it is often the case that yes, the dependent
variable is more or less a linear or log-linear function of several
variables. If such descriptions do no violence to the reality from
which the data was culled, then OLS is the method to use, and using
more general techniques will not be any more persuasive. But if these
assumptions are not true, then using OLS is at best unpersuasive, and at
worst disingenuous. We have the computing power to write down a model of
the real world in arbitrary detail, and describe and test that model in
reasonable time.

Thus, having covered the special cases in Chapters \ref{projections}
and \ref{gauss}, Chapter \ref{mle} will show you how to write down the
likelihood functions to describe your model, as in Probit or Normit
estimations, how to find their maxima, and how to test hypotheses using
a likelihood ratio test.

Sometimes, even a likelihood function does not do justice to the model
you are attempting to describe, in which case the best thing to do is
simply write down the process and do a few million random draws from
that process and see what properties those draws have.
Chapter \ref{boot} will cover Monte Carlo methods for such general
model description, including bootstrapping and jackknifing, which are
techniques to find variances from data when the classic Central Limit
Theorems fail.  
\index{Ordinary Least Squares|)}

\paragraph{\treesymbol Seeing the forest from the trees} Statistics textbooks
must tread a fine line. On the one hand, they must be didactic texts,
explaining concepts to the reader, while on the other they must be a
catalog of methods to apply should your data look just so.

Sections marked with a \ind{\treesymbol} cover details that may be
skipped on a first reading. They are not necessarily advanced
in the sense of being somehow more difficult than unmarked text, but a
reader who is trying to get the overall picture may want to cover them 
later. 

\comment{
\section{Outline} 
Part I covers general scientific computing, and will be valuable to
anyone who needs to program a computer to handle and do math with large volumes of data.
Since I am assuming that you are computer-literate but
not a C programmer, Chapter \ref{c_crash} will give you a crash course
in C. It will not only get you familiar with the rules of the language,
but how to best think about problems in C. 
Chapter \ref{sql} will give you a quick overview of the second language,
SQL, which is a language that facilitates
producing tables of just the right form from a dataset. C and SQL are
excellent complements for data analysis: things which are very difficult
in one are often trivial in the other.
Chapter \ref{linear_algebra}
will then introduce you to the package of C functions most useful for
doing stats: the GNU Scientific Library (GSL). Notably, it will cover
how to do linear algebra using the GSL. Chapter \ref{apop} covers
Apophenia, a library of functions that bind together the GSL and SQLite
to facilitate data handling and statistics. Finally, Chapter
\ref{gnuplot} will cover creative uses of Gnuplot to display data.


Given those tools, coding statistics will be easy. Part II includes a chapter
devoted to each of the three categories of statistics above: Chapter \ref{projections}
handles the process of describing your data, using simple calculations
or projection techniques; Chapter
\ref{gauss} covers methods of comparing your data to various Gaussian
distributions, such as the t test, F test, and chi-squared test; Chapter
\ref{mle} will show you how to write down your likelihood functions and
find their maxima, as in Probit or Normit estimations, and how to test
hypotheses using a likelihood ratio test. Chapter \ref{boot} will cover
Monte Carlo methods such as 
bootstrapping and jackknifing, which are dirty tricks which will one
day save your life.  
}


\comment{
\paragraph{Acknowledgments} Thanks to Ms. Annjeanette Agro for
graphic design suggestions.}

