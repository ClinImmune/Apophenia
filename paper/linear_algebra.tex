\startonecol \chapter[Linear Algebra]{Linear algebra} \label{linear_algebra} \endonecol

Recall the philosophy of C: the language provides only the most basic of
basics, such as addition and division, and everything else one would
want to do is provided by a function library. Most of statisical analysis is
described as operations on matrices and vectors, so the first key
extension a mathematician needs is a package to do basic vector and
matrix maintenance and linear algebra.

\index{GSL|see{GNU Scientific Library}}
There are many on the market; this chapter uses the \ind{GNU Scientific
Library} (GSL). The GSL is recommended because it is actively supported
and will work on about as many platforms as C itself. Apophenia is built
on the GSL, and many of its structures have GSL matrices or vectors at
their core, so low-level operations on \cinline{apop\_data} sets become
\cinline{gsl\_matrix} operations.

\section{The GSL's matrices and vectors}
As you saw in Chapter \ref{c_crash}, arrays and matrices can be directly implemented in C, but for
the rest of the book, I'll be sticking to the GSL's matrix and vector objects.
If you like using raw arrays better, it's easy to switch back and forth; see
section \ref{asst_conversions}.


\cindex{gsl\_matrix}
\cindex{gsl\_matrix\_set}
\cindex{gsl\_matrix\_set\_row}
\cindex{gsl\_matrix\_set\_col}
\cindex{gsl\_matrix\_alloc}
\cindex{gsl\_matrix\_get}
\cindex{gsl\_vector\_alloc}
\cindex{gsl\_matrix\_free}
\cindex{gsl\_vector\_free}
\index{declaration!of gsl\_matrix@of \cinline{gsl\_matrix}}
\index{declaration!of gsl\_vector@of \cinline{gsl\_vector}}
\index{for@\cinline{for}!example of}
Here is a complete program that will do a few useless things to a few sample
objects:\label{gslexample}
\begin{lstlisting}
#include <gsl/gsl_matrix.h>
#include <stdio.h>

int main(void){
   gsl_matrix   *m = gsl_matrix_alloc(10,10);
   gsl_vector   *v = gsl_vector_calloc(10);
   int  i;

   for (i=0;i< m->size1; i++){
      gsl_matrix_set(m, i, 0, i) ;
   }
   printf("Here is point (3,0): %g\n", gsl_matrix_get(m, 3,0));
   gsl_matrix_set_row(m, 3, v);
   printf("Here is point (3,0) again: %g", gsl_matrix_get(m, 3,0));
   gsl_matrix_free(m);
   gsl_vector_free(v);
   return 0;
}
\end{lstlisting}
\paragraph{A walk through the code}
Here is what just happened: we allocated a 10$\times$10 matrix and a vector of
length 10.  For the sake of variety, we  allocated the two differently.
\cinline{gsl\_matrix\_alloc} simply set aside a block of memory for the matrix,
and that block may have garbage in it. Meanwhile, \cinline{gsl\_vector\_calloc} set
aside some space for the vector \cinline{v}, and set all the values of \cinline{v} to
zero.  We were able to do these allocations in the declaration itself.

That done, the \cinline{for} loop put some values in the first column of the matrix (leaving garbage in the rest of the matrix). 
The syntax should be familiar to you from subsection \ref{for_loops}: we start at
zero, not one, and increment up to the size of the matrix, which in this case is
\cinline{m-$>$size1}. You will recognize this as accessing a structure, which is exactly
what we are doing: the declaration \cinline{gsl\_matrix *m} means that
\cinline{m} is a pointer to a \cinline{gsl\_matrix}, which is a structure
whose definition you can look up if you are so inclined. If you do, you
will see that it includes elements \cinline{size1} and \cinline{size2},
for the row and column sizes of the matrix. [Row always comes first,
then Column, just like the order in Roman Catholic, Randy Choirboy,
or RC Cola.] Since the vector has only one dimension, the analagous
element of the vector structure is \cinline{v-$>$size}.

Next, we copied the vector to the third row of the matrix using \cinline{gsl\_\-ma\-trix\_\-set\_\-row(m, 3, v)}. Notice that in so doing, we
overwrote the three at the point (3,0) of the matrix with a zero from
the third element of \cinline{v}.

Finally, we freed the memory used for the vectors. This is not 
necessary for a small program,
but it's a good habit to get into for
when you start getting the monolithic analyses, which you may not be
able to run on your PC if you don't keep the memory clear of debris.

\paragraph{Naming conventions}  \index{naming functions}
Every function in the GSL library will begin with \cinline{gsl\_}, and
the first argument of all of these functions will be the object to be acted upon.
Every function which affects a matrix will begin with \cinline{gsl\_matrix\_}
and similarly with vectors and their functions, which all begin with \cinline{gsl\_vector\_}. 
Apophenia generally sticks to this as well, and 100\% of its functions
begin with \cinline{apop\_} and a great majority of them begin with a data
type such as \cinline{apop\_data\_}, \cinline{apop\_est\-i\-mate\_}, \&c.

The consistency of the naming means that you have more to type, but
less to memorize. Given the package-object-verb form, you can likely
guess the name of the function you want. Your text editor probably has
some sort of name completion command, which you may want to look up. E.g.,
vim users, try $<$ctrl-n$>$ after typing a few characters.

Another alternative is to start writing your own functions. For example, you
could write a file \cinline{my\_convenience\_fns.c}, which could include:

\cindex{gsl\_matrix\_get}
\cindex{gsl\_matrix\_set}
%\begin{verbatim}
\begin{lstlisting}
double mget(gsl_matrix *m, int row, int col){
   return gsl_matrix_get(m, row, col);
}

double vget(gsl_vector *v, int row){
   return gsl_vector_get(v, row);
}
\end{lstlisting}
%\end{verbatim}

You will also need a header file, \cinline{my\_convenience\_fns.h}:
%\begin{verbatim}
\begin{lstlisting}
double mget(gsl_matrix *m, int row, int col);
double vget(gsl_vector *v, int row);
\end{lstlisting}
%\end{verbatim}

After throwing an \cinline{\#include "my\_convenience\_fns.h"} at the top of your
program, you will be able to use your abbreviated syntax such as \cinline{mget(v,3)}.
It's up to your \ae sthetics as to whether your code will be more or less
legible after you make these changes, but you will need to remember to
bring your convenience functions with you everywhere you go.



\section{The \ind{BLAS}} 
\index{matrices!dot product|(} \index{dot product|(}
Before there was the GSL, there was the BLAS---the basic linear algebra
system. The GSL has a few functions to interact with the BLAS. In fact,
it has 86. Here are the three that you will actually 
use.\footnote{The names of the functions here fit in with the system
for the other 83 functions you won't ever use. They are a combination
of D=double precision, GE=general, M=matrix, V=vector.}

\paragraph{Matrix $\cdot$ vector} Here is the function you will use to calculate the dot product of a
matrix and a vector:
\cindex{gsl\_blas\_dgemv|(}
\begin{lstlisting}
int gsl_blas_dgemv (CBLAS_TRANSPOSE_t TransX, float alpha, 
          const gsl_matrix * X, const gsl_vector * v, 
          float beta, gsl_vector * y)
\end{lstlisting}

This will put into the vector $y$ either the value 
$\alpha X v + \beta y$ or $\alpha X' v + \beta y$, as determined by the
first argument.
\cinline{CBLAS\_TRANSPOSE\_t} is a type defined by the GSL that can take
one of two values: \cinline{CblasNoTrans} and \cinline{CblasTrans}. If
it takes the former value, then $X$ is left alone;
if it takes the latter, it is transposed to $X'$.

To give a concrete example, assume you have already got some vectors and matrices which have the following
declarations:
\begin{lstlisting}
gsl_vector *beta, *gamma;     
gsl_matrix *x, *y;           
\end{lstlisting}

Then, to calculate $X\cdot \beta$, we'd need:

\begin{lstlisting}
gsl_vector *beta_dot_x      = gsl_vector_calloc(x->size1);
gsl_blas_dgemv (CblasNoTrans, 1.0, x, beta, 0.0, beta_dot_x);
\end{lstlisting}

Notice that we used \cinline{calloc}, instead of just \cinline{alloc}, because
the system will add $x\beta$ to \cinline{beta\_dot\_x}, not just write it in,
so \cinline{beta\_dot\_x} needs to start as all zeros.
\cindex{gsl\_blas\_dgemv|)}

\paragraph{Vector $\cdot$ vector}
To find the dot product of two vectors, use this function:
\cindex{gsl\_blas\_ddot}
\begin{lstlisting}
int gsl_blas_ddot (const gsl_vector * x, const gsl_vector * y, double * result);
\end{lstlisting}

For example,

\begin{lstlisting}
double *beta_dot_gamma;
gsl_blas_ddot (beta, gamma, beta_dot_gamma);
\end{lstlisting}

\paragraph{Matrix $\cdot$ matrix}
Finally, to take the dot product of two matrices, use:
\cindex{gsl\_blas\_dgemm|(}
\begin{lstlisting}
int gsl_blas_dgemm (CBLAS_TRANSPOSE_t TransX, 
    CBLAS_TRANSPOSE_t TransY, 
    double alpha, const gsl_matrix * X, const gsl_matrix * Y, 
    double beta, gsl_matrix * dot_product);
\end{lstlisting}
which will calculate \cinline{dot\_product} $= \alpha op(X) op(Y) + \beta$ \cinline{dot\_product}. $op(X)$ and
$op(Y)$ will be either the matrix or its transpose, as above, depending on whether you choose \cinline{CblasTrans}
or \cinline{CblasNoTrans}. For example, here is $X'Y$:

\begin{lstlisting}
gsl_matrix *x_dot_y      = gsl_matrix_calloc(x->size2, y->size2);
gsl_blas_dgemm (CblasTrans,CblasNoTrans, 1, x, y, 0, x_dot_y);
\end{lstlisting}

To give a more extended example, let us say that we have a \ind{transition matrix}, showing whether one can go from a row state to a
column state. Omitting the labels for now, here is a sample:
\label{twostep}\lstinputlisting{sources/markov_data} 
[This is not really a \ind{Markov matrix}, because its
columns are not normalized to one; the reader will see below that it
still is a valid description of a set of transitions.]

Here is a complete program to read in this matrix and then output this
count of two-step transitions. The steps should be clear: read the data,
allocate a space for the output, calculate the dot product, and print
the result. It introduces a new data type, \cind{apop\_data} to be
discussed fully in Chapter \ref{apop}. For now, just take it to be a
\cinline{gsl\_matrix} with labels.
\lstinputlisting{sources/markov.c}
Here is the output:
\begin{lstlisting}
    2       3       1       1       2       3
    3       4       3       3       3       4
    2       2       3       3       2       2
    3       4       3       3       3       4
    4       5       4       4       4       5
    3       4       3       3       4       4
\end{lstlisting}
This tells us, for example, that there are three ways to transition from state one to
state two in two steps (1 $\Rightarrow$ 1 $\Rightarrow$ 2, 
1 $\Rightarrow$ 2 $\Rightarrow$ 2,  and 1 $\Rightarrow$ 6 $\Rightarrow$
2).

\exercise{\label{quadq}The quadratic form $\Xv'\Yv\Xv$ appears very frequently in
statistical work. Write a function with the header
\cinline{gsl\_matrix *quadratic\_form(gsl\_matrix *x, gsl\_matrix *y);} 
that takes two \cinline{gsl\_matrix}es and returns the quadratic form as above. Be sure to check that \cinline{y} is square and has the same dimension as \cinline{x-$>$size1}.}

\cindex{gsl\_blas\_dgemm|)}

\index{matrices!dot product|)}
\index{dot product|)}
\paragraph{Scalars} The GSL provides predictable functions for
interacting vectors or matrices with scalars. Here is code to add one to
and then double all the elements of a matrix and a vector:
\begin{lstlisting}
gsl_matrix_add_constant(a_matrix, 1);
gsl_matrix_scale(a_matrix, 2);

gsl_vector_add_constant(a_vector, 1);
gsl_vector_scale(a_vector, 2);
\end{lstlisting}
\cindex{gsl\_matrix\_add\_constant} \cindex{gsl\_matrix\_scale}
\cindex{gsl\_vector\_add\_constant} \cindex{gsl\_vector\_scale}

\section{Matrix inversion and equation solving}  \index{matrices!inversion}\index{matrices!determinants}
Matrix inversion is one of the most computationally intensive problems
around. In fact, some will tell you it is the problem for which computers were invented.

\codefig{invertmatrix}{A function that inverts a matrix using via an LU decomposition}

The GSL does not include a function to directly invert a general matrix:
instead, the user has to go through a triangular decomposition. Figure
\ref{invertmatrix} shows a function to do all of the steps for us.
Examples for using this function are located throughout the book; for
example, see the calculation of OLS coefficients on page \pageref{ols}.

Figure \ref{invertmatrix} is cut and pasted from the Apophenia source
code; \cind{apop\_det\_and\_inv} does indeed invert and calculate the
determinant of an input matrix. This is what \ttind{apop\_OLS} will use if
you ask for covariances.

\index{Householder solver}
Because inversion is so computationally intensive, you are better off
not doing so.  For example, we often write the OLS parameters as $\beta
=(X'X)^{-1}(X'Y)$, but you could implement this as solving
$(X'X)\beta = X'Y$, which involves no inversion. If \cinline{xpx} is the
matrix $X'X$ and \cinline{xpy} is the vector $X'Y$, then 
\cind{gsl\_linalg\_HH\_solve}\cinline{(xpx, xpy, out-$>$parameters)} will
return $\beta$. This is how \cinline{apop\_OLS} will find $\beta$ if it
is all you ask for.


\section{Shunting data} \label{asst_conversions} 
There are many facets to dealing with a computer that one would call
tedious, but for me, the most tedious of all is converting data formats.
You can describe the process easily, say `pull the data and find
$\Xv'\Xv$', but the data is in a text file and your function to take dot
products accepts \cinline{gsl\_vectors}, and going from one
to the other will take you much more time than calling the function in
the end.

To that end, this section presents
suggestions for converting among the various data formats used in this
book. It is not an exciting
read, and outside of Method V, there is little new material; you may
prefer to take this section as a reference to refer to as necessary. Table
\ref{conversiontab} provides the key to the method most appropriate for
each given conversion. 

From/to pairs marked with a dot are
left as an exercise for the reader; none are particularly difficult, but
may require going through another format; for example, one can go from a
\cinline{double[ ]} to an \cinline{apop\_data} set via \cinline{double[
]} $\Rightarrow$ \cinline{gsl\_matrix} $\Rightarrow$ \cinline{apop\_data}.
One can go from any format to any other in two steps, as proven above
using the transition matrix on page \pageref{twostep}. 

\def\rcap#1{\rotatebox{45}{#1}\hskip -15pt }
\def\rcapc#1{\rotatebox{45}{\cinline{#1}}\hskip -15pt }
\def\cd{$\cdot$}
\begin{figure} \begin{center}
\begin{tabular}{ll}
\rotatebox[x=15mm]{90}{From}&
\begin{tabular}{lp{0.4cm}p{0.4cm}p{0.4cm}p{0.4cm}p{0.4cm}p{0.4cm}}
   &   &&&To\\
    & \rcap{Text file} & \rcap{Db table} & \rcapc{double[]} 
        & \rcapc{gsl\_vector} & \rcapc{gsl\_matrix}& \rcapc{apop\_data}\\
Text file               & C & F & \cd & \cd & \cd &  F                   \\
Db table                & \cd & Q & \cd & Q & Q & Q                  \\
\cinline{double[ ]}      & \cd & \cd & C & F & F & \cd                  \\
\cinline{gsl\_vector}   & P  & P & F & C & \cd & F                    \\
\cinline{gsl\_matrix}   &  P & P & F & V &C &  F               \\
\cinline{apop\_data}    & P & P & F & \cd & S &C
\end{tabular}
\end{tabular}
\caption{Methods of conversion} \label{conversiontab}
\end{center}\end{figure} 


\paragraph{Method C: Copying} There are two styles of copying.
One is the family of functions with \cind{memcpy} in the name, that are
of the form \cinline{memcpy(dest, src)}. The assume that \cinline{dest}
is already allocated.  
Apophenia provides functions to allocate and copy in one step, with
names of the form \cinline{apop\_...\_copy}.

\cindex{gsl\_vector\_memcpy} \cindex{gsl\_matrix\_memcpy}
\cindex{apop\_data\_memcpy} \cindex{apop\_data\_memcpy} \cindex{system}
\lstset{texcl=true} %For this whole section.
\begin{lstlisting}
//Text file $\Rightarrow$ Text file
//Just use the system's file copy command.
    system("cp %s %s", from_file_name, to_file_name);
//\cinline{gsl\_vector} $\Rightarrow$ \cinline{gsl\_vector}
    gsl_vector *copy   = gsl_vector_alloc(original->size);
    gsl_vector_memcpy(copy, original);
    gsl_vector *copy2 = apop_vector_copy(original);
//\cinline{double[ ]} $\Rightarrow$ \cinline{double[ ]}
//You need to know the size of the original.
    double *copy1 = malloc(sizeof(double) * original_size);
    mecpy(copy1, original, sizeof(double) * original_size);
    double *copy2[original_size];
    mecpy(copy2, original, sizeof(double) * original_size);
//\cinline{gsl\_matrix} $\Rightarrow$ \cinline{gsl\_matrix}
    gsl_matrix *copy   = gsl_matrix_alloc(original->size1, original->size2);
    gsl_matrix_memcpy(copy, original);
    gsl_matrix *copy2 = apop_matrix_copy(original);
//\cinline{apop\_data} $\Rightarrow$ \cinline{apop\_data}
    apop_data *copy1 = apop_data_alloc(original->matrix->size1, original->matrix->size2);
    apop_data_memcpy(copy1, original);
    apop_data *copy2 = apop_data_copy(original);
\end{lstlisting}

\paragraph{Method F: Function call}
\cindex{apop\_vector\_to\_data}\cindex{apop\_matrix\_to\_data}
\cindex{apop\_text\_to\_data} \cindex{apop\_text\_to\_db}
\cindex{apop\_array\_to\_matrix} \cindex{apop\_array\_to\_vector}
\cindex{apop\_line\_to\_matrix} \cindex{apop\_line\_to\_data}
\index{arrays}

These are functions designed to convert one format to another.

There are two ways to express an array of \cinline{double}s. One is to
declare a pointer to pointers to \cinline{double}s, and the other is to
delcare double-subscripts:
\begin{lstlisting}
double **method_one = malloc(sizeof(double*)*size_1);
double method_two[size_one][size_two] = {{2,3,4},{5,6,7}};
\end{lstlisting}
The second method seems convenient, because it lets you allocate the
matrix at once. But due to minuti\ae{} that will not be discussed here
(see \citet[p 113]{kandr:c}), method two is too much of a hassle to
be worth anything. 

Instead, declare your data as a single line, listing the entire first
row, then the second, et cetera, with no intervening brackets. Then, use
the \cinline{apop\_line...} functions to convert to a matrix. Below, the
first set of examples convert from an array decared as in method one
above, and the second set shows an array declared as a single line and
then converted to a $2 \times 3$ matrix.
\begin{lstlisting}
//text $\Rightarrow$ db table
//The first number states whether the file has row names; the second
//whether it has column names. Finally, if no colnames are present,
//you can provide them in the last argument as a \cinline{char **}
    apop_text_to_db("original.txt", "tablename", 0 , 1, NULL);
//text $\Rightarrow$ \cinline{apop\_data}
    apop_data *copy = apop_text_to_data("original.txt", 0 , 1);
//\cinline{*double} $\Rightarrow$ \cinline{gsl\_vector}
    gsl_vector  *copy = apop_array_to_vector(original, original_size);
//\cinline{**double} $\Rightarrow$ \cinline{gsl\_matrix}
    gsl_matrix *copy = apop_array_to_matrix(original, original_size1, original_size2);

//\cinline{double[ ]} $\Rightarrow$ \cinline{gsl\_matrix}
    double original[] = {2,3,4,5,6,7};
    gsl_matrix *copy = apop_line_to_matrix(original, original_size1, original_size2);
//\cinline{double[ ]} $\Rightarrow$ \cinline{apop\_data}
    apop_data *copy = apop_line_to_data(original, original_size1, original_size2);

//\cinline{gsl\_vector} $\Rightarrow$ \cinline{double[ ]}
    double *copy;
    int copy_size   = original_vec->size;
    apop_vector_to_array(original_vec, &copy);
//\cinline{gsl\_vector} $\Rightarrow$ \cinline{apop\_data}
    apop_data   *copy = apop_vector_to_data(original_vec);
//\cinline{gsl\_matrix} $\Rightarrow$ \cinline{apop\_data}
    apop_data   *copy = apop_matrix_to_data(original_matrix);
\end{lstlisting}

\paragraph{Method P: Printing}
\cindex{apop\_vector\_print} \cindex{apop\_matrix\_print} \cindex{apop\_data\_print}
Apophenia's printing functions will print to screen, text file, or
database, depending on the value of \cind{apop\_opts.output\_type}.
\begin{lstlisting}
//\cinline{gsl\_vector} $\Rightarrow$ text file
//\cinline{gsl\_matrix} $\Rightarrow$ text file
//\cinline{apop\_data} $\Rightarrow$ text file
    apop_opts.output_type = 't'
    apop_vector_print(original_vector, "text_file_copy");
    apop_matrix_print(original_matrix, "text_file_copy");
    apop_data_print(original_data, "text_file_copy");
//\cinline{gsl\_vector} $\Rightarrow$ db table
//\cinline{gsl\_matrix} $\Rightarrow$ db table
//\cinline{apop\_data} $\Rightarrow$ db table
    apop_opts.output_type = 'd'
    apop_vector_print(original_vector, "db_copy");
    apop_matrix_print(original_matrix, "db_copy");
    apop_data_print(original_data, "db_copy");
\end{lstlisting}


\paragraph{Method Q: Querying}
\cindex{apop\_query} \cindex{apop\_query\_to\_vector}
\cindex{apop\_query\_to\_matrix} \cindex{apop\_query\_to\_data}
The only way to get data out of a database is to query it out. Apophenia
provides a function that will produce the appropriate output.

\begin{lstlisting}
//db table $\Rightarrow$ db table
    apop_query("create table copy as \
    select * from original");
//db table $\Rightarrow$ \cinline{gsl\_vector}
    apop_query_to_vector("select * from original");
//db table $\Rightarrow$ \cinline{gsl\_matrix}
    apop_query_to_matrix("select * from original");
//db table $\Rightarrow$ \cinline{apop\_data}
    apop_query_to_data("select * from original");
\end{lstlisting}


\paragraph{Method S: Subelements} Sometimes, even a function is just
overkill; you can just pull a subelement from the main data item.

Notice, by the way, that the \cinline{data} subelement of a
\cinline{gsl\_vector} can not necessarily be copied to a
\cinline{double[ ]}---the \airq{stride} may be wrong; see the GSL manual
for details. Instead, use the copying functions from Method F above.

\begin{lstlisting}
//\cinline{apop\_data} $\Rightarrow$ \cinline{gsl\_matrix}:
    my_data_set -> matrix
\end{lstlisting}
\lstset{texcl=false}

%\eject

\comment{
\marginaliafixed{18}{Allocating and using at once}{    \index{declaration}
One slight convenience that helps with the annoyance of shunting data from
one type to another is that you can declare a variable and assign it at the
same time. Let us say that you have already filled a \cinline{gsl\_matrix
*m} with data and want it to be an \cinline{apop\_data} structure. Then
you can declare and allocate your new structure with one line:\\
%\begin{lstlisting}
\cinline{apop\_data   *mdata  = apop\_matrix\_to\_data(m);}\\
%\end{lstlisting}
On the left-hand side, we are allocating an \cinline{apop\_data *} pointer, and
on the left-hand side, \cind{apop\_data\_from\_matrix} returns an 
\cinline{apop\_data *} pointer. As in the example on page \pageref{gslexample}, the same can be
done with any function with \cinline{alloc} in the name, because they will also
return a pointer to the appropriate type.
}
}

\paragraph{Method V: Views}\index{views}\index{matrices!views}
The GSL includes a convenient structure for pulling a vector from a
matrix. Here is how to get the fifth row of \cinline{a\_matrix} into a vector view:

\begin{lstlisting}
gsl_vector v;
v = gsl_matrix_col(a_matrix, 4).vector;
apop_vector_show(&v);
\end{lstlisting}
[Notice that this function is named \cinline{col}, not \cinline{column}.]
For the fifth row, use \cinline{gsl\_matrix\_row(a\_\-ma\-trix, 4)}. 
\cindex{gsl\_matrix\_row} 

The vector view is a nice test of your comprehension of the details of
pointers and memory allocation from Section \ref{pointers}.
\cinline{gsl\_ma\-trix\_col} returns a
\cind{gsl\_vector\_view}---not a vector, but the actual
structure.  Inside the \cinline{gsl\_vec\-tor\_view}, you will find 
an element named \cinline{vector}, which is again not a pointer but an
automatically allocated structure. Inside that \cinline{.vector}
element, one would find a pointer to the data.

When the function ends, all of the automatically-allocated data is
thrown out, including \cinline{gsl\_\-vec\-tor\_\-view} and its
\cinline{vector} element, but the data itself is untouched.

Notice further that every function that handles a
\cinline{gsl\_\-vec\-tor} actually takes a {\em pointer} to a 
\cinline{gsl\_\-vec\-tor}. Thus, the fact that you have an actual 
\cinline{gsl\_\-vec\-tor} on your hands means that you will have to pass
its address to the various functions, as in the call to
\cinline{apop\_\-vec\-tor\_\-show} above.
\cindex{apop\_vector\_show}

If you want to retain the vector after the function exits, your best bet
is to just copy it to another vector:

\cindex{gsl\_vector\_alloc}
\cindex{gsl\_vector\_memcpy}
\cindex{gsl\_matrix\_col}
\begin{lstlisting}
gsl_vector *a_new_vector = gsl_vector_alloc(a_matrix->size1);
gsl_vector v = gsl_matrix_col(a_matrix, 4).vector;
gsl_vector_memcpy(a_new_vector, &v);
\end{lstlisting}


\section{\treesymbol Numbers} \index{infinity} \index{NaN|see{not a number}} \index{not a number} \cindex{GSL\_NEGINF} \cindex{GSL\_POSINF} \cindex{GSL\_NAN} \label{numbers}
Floating-point numbers (both \cind{float} and \cind{double}) can take three special values: \cinline{Inf}, \cinline{-Inf}, and \cinline{NaN}. Data handlers will mostly be
interested in \cinline{NaN} (read: not a number), which is an appropriate way to represent missing data. Here is a
program to show you the syntax. All four \cinline{if} statements will be true, and print their associated
statements.
\cindex{gsl\_finite}
\cindex{gsl\_isnan}
\cindex{gsl\_isinf}
\begin{lstlisting}
#include <gsl/gsl_math.h>   //NaN handlers
#include <stdio.h>          //printf

int main(){
double missing_data = GSL_NAN;
float big_nuber = GSL_POSINF;
float negative_big_number = GSL_NEGINF;
if (gsl_isnan(missing_data))
    printf("I'm missing a data point.");
if (gsl_finite(big_number)== 0)
    printf("big_number is not finite.");
if (gsl_finite(missing_data)== 0)
    printf("missing_data isn't finite either.");
if (gsl_isinf(negative_big_number)== -1)
    printf("negative_big_number is negative infinity");
return 0;
}
\end{lstlisting}

Since floating-point nubers can take these values, division by zero
won't crash your program. \cinline{float f = 1.0/0.0} will result in
\cinline{f == GSL\_POSINF} and \cinline{f = 0.0/0.0} will result in \cinline{f ==
GSL\_NAN}. However, integers have none of these luxuries: run \cinline{int
i = 1/0} and you will get \cind{Arithmetic exception (core dumped)}.

