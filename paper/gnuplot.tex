\chapter{Gnuplot} \label{gnuplot}

This chapter will give you some tips on plotting using Gnuplot, which is a graphing package which is as
open and freely available as gcc. [It is unrelated to the GNU, by the way. The name is a compromise
between the names the two main authors preferred: nplot and llamaplot.]

Gnuplot does nothing but plot points. It has some interface with the GSL to plot functions for you, but
you mostly won't be concerned with those: you'll just want to plot your data.


\section{Dumping output to an agreeable format} 

\subsection{Histograms} Gnuplot, refusing to do calculations for you, makes making histograms a pain.
Fortunately, the GSL has you covered, with the \ttind{gsl\_histogram} object.

\section{Autogenerating Gnuplot scripts} 
Since there are so many ways to tweak Gnuplot, I like to include a subprocedure to write Gnuplot scripts
in the main program. I'll give a few examples here.

