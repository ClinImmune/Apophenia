\chapter{Gnuplot} \label{gnuplot}
%@-     so lgrind will play nice with makeindex
\index{Gnuplot|(}
\index{plotting|see{Gnuplot}}
\index{graphing|see{Gnuplot}}

This chapter will give you some tips on plotting using Gnuplot, a graphing
package that is as open and freely available as gcc.\footnote{There is
some politics about how this is not strictly true: the maintainers of
Gnuplot will not allow you to modify their code and then distribute the
modified package independently (i.e., to \airq{fork} the code base). The project is entirely unrelated to the
GNU, and the name is simply a compromise between the names that the two
main authors preferred: nplot and llamaplot.}

\paragraph{The language} Gnuplot does nothing but plot points and
lines. As such, the language only has one verb---\cinline{plot}---with
a few variants (\cinline{splot}, \cinline{replot}). A Gnuplot command
file generally consists of a long string of \cinline{set} commands to
indicate what the axes should look like, the point and line formats,
et cetera, and a final \cinline{plot} command to do the actual work.


\section{From arrays to plots} \label{gnuprint}
The line \cinline{plot '-';} tells Gnuplot to plot the (2-D) data to come. The
command for a 3-D plot is \cinline{splot '-'}. That means that the process
of turning a matrix into a basic plot is trivial. Let \cinline{data} be an
\cinline{apop\_data} set whose first and fifth columns we would like to plot
against each other. Then we need to create a file, put a \cinline{plot
'-';} command in the first line, and then fill the remainder with the
data to be plotted.  
\begin{lstlisting}
FILE    *f = fopen("plot_me", 'w');
fprintf("plot '-' using 1:5;\n");
fclose(f);
apop_data_print(data, "plot_me");
\end{lstlisting}
The second option to \cind{fopen} should be either \cinline{'a'} for append or
\cinline{'w'} for write anew; in this case, we want to start with a clean file so we use \cinline{'w'}.
Notice that \cind{apop\_data\_print} defaults to appending; you
would need to set \cinline{apop\_opts.output\_append = 0} to overwrite.
\cindex{apop\_opts.output\_append}

The clause \cinline{using 1:5} in the plot command tells Gnuplot to use
only the first and fifth columns of the data set---so there is no need to pare
down the data set in memory. Notice that the first column in Gnuplot is
1, not 0. At the end of this, \cinline{print\_me} will be executable by
Gnuplot. There are a few ways to get Gnuplot to read the file from the
typical UNIX-style command prompt:
\begin{lstlisting}
gnuplot -persist < plot_me
(or)
gnuplot plot_me -
\end{lstlisting}
The first option will open a window and leave the command line inactive
until you close the window;
the second will give you a Gnuplot command line that you will have to
manually exit
from (by typing \cinline{exit} or $<$ctrl$>$-D, or you can send in more
Gnuplot commands). On many systems, you can
interact with the plot, spinning 3-D plots with the mouse or selecting
subregions of 2-D plots to zoom into.

\subsection{Histograms} Gnuplot, refusing to do calculations for you,
makes making histograms a pain.  Fortunately, the GSL has you covered,
with the \cind{gsl\_histogram} object, which first made its appearance
in Section \ref{randomnumbers}, when we needed to generate PDFs from data.

It creates a set of bins, and
then you can call \cinline{gsl\_histogram\_increment} to put
data into the right bin. Below is the code for \cind{apop\_plot\_hist\-o\-gram}. 
It allocates a histogram with
\cinline{bin\_ct} bins, then sets its range to just past the min and max
of the data. Then the for loop pulls data points from the vector and
dumps them to the histogram, without any further calculation on our part.
Once everything is neatly aggregated, the function opens a file and dumps the data to a Gnuplottable file.
\begin{lstlisting}
void apop_plot_histogram(gsl_vector *data, size_t bin_ct, char *outfile){
int             i;
FILE *          f;
double          min, max, pt;
gsl_histogram   *h      = gsl_histogram_alloc(bin_ct);
    gsl_vector_minmax(data, &min, &max);
        gsl_histogram_set_ranges_uniform(h, min-GSL_DBL_EPSILON, max+GSL_DBL_EPSILON);
    for (i=0; i < data->size; i++){
        pt  = gsl_vector_get(data, i);
        gsl_histogram_increment(h, pt);
        }

    //Now that you have a histogram, print it.
        if (outfile == NULL)    f       = stdout;
        else            f       = fopen(outfile, "a");
    fprintf(f, "set key off                                     ;\n\
                        set style data histograms               ;\n\
                        set style histogram cluster gap 0       ;\n\
                        set xrange [0:%i]                       ;\n\
                        set style fill solid border -1          ;\n\
                        set boxwidth 0.9                        ;\n\
                        plot '-' using 2:xticlabels(1);\n", bin_ct);
    for (i=0; i < bin_ct; i++)
    fprintf(f, "%4f\t %g\n", h->range[i], gsl_histogram_get(h, i));
    if (outfile !=NULL)    fclose(f);
}
\end{lstlisting}
\index{Gnuplot|)}
