
[First, let me warn you that this book is very incomplete. The first three chapters are finished, and it
tapers off from there, to the point that the last chapter is just a placeholder. I feel that it should
already have utility to many people, so it is here for your perusal.]
\vskip .5cm

This book is a write-up of my experience writing statistical analyses.
The methods I advocate go against the current orthodoxy, as described
in full-color advertisements in leading journals and magazines, so I want to
use this preface to be especially explicit in explaining why I eschew
the statistics packages that others love so much.

\comment{
There are two reasons I went through the effort of writing down my experiences in
complete sentences.

The first is that I want more people to take an interest in my methods of
computing. More people working like I do means more support for me, in terms of
software available, debugging, critique, and people who don't look at me funny
when I describe my methods. 
}

\index{statistics packages!rants regarding|(}
Here is what I look for in software, based on many years of experience with a large range of programs. I hope you agree that all of these features are desirable: \\
$\bf 1.$ It should be extensible and flexible: all arbitrary limits are bad.  \\
$\bf 2.$ It should be portable: I should be able to sit down at a computer at work, at home, or while travelling,
and work on my projects.\\
$\bf 3.$ It should be transparent: if something breaks I know if it's
my fault or the software's fault, and can fix it in either case.

Conversely, here is the list which many people go by:\\
$\bf 4.$ It should be easy to sit down and use immediately.\\

Ease-of-initial-use is undeniably a good thing---but if it is at the cost
of items 1--3, then the software which gave you an initial jump-start will
bog you down over the course of weeks, months, or years. There is
undoubtedly a trade-off: the designers of the software can hide details
from the user so that the user won't have to bother learning them---but
the hidden details very frequently come at the cost of \#1 and \#3. Some
programs have beautiful graphical interfaces, but these are very difficult
to program without a specific machine in mind, so \#2 gets thrown out.

\paragraph{The long run}
I have certainly benefited from easy-to-learn software, because I've had to learn
so many programming languages and so many stats packages. Each time, I thought
this would be the one I could do all my work in from now on, and each time, I
found that there were arbitrary limits and I had to start over with yet another software package.
Sometimes this was because I ran into failures of \#3 (transparency): the package
didn't do what I'd expected, and after several hours of trying to work
out why, I either failed to determine it or found that there was a bug
in the code which I could do nothing about. The value of \#2 (portability) hit me
when I started a new job, and the thousand lines of code I had written on a Unix-based version of R wouldn't run
on the copy of R on the Windows machine at work. I do work in a field for which the
assumptions of OLS frequently fail, so failures in \#1 (extensibility) hit me often and hard,
as discussed below. Other researchers have told me many a sob story
about how they designed the perfect method around a small sample, and
then their stats package couldn't handle the full data set, due to
arbitrary limitations hard-coded by the designers.

You should ask yourself where you will be in five years. If you expect that you will never do
anything more difficult than a linear regression on well-formed IID data,
or if you are trying to slog through your department's stats requirement
so you can never look at another data set again, then by all means,
use the easiest stats package you can get away with. But if you expect to
be doing statistical work for a larger part of your career, then using a
stats package which skimps on criteria \#1-\#3 for the sake of \#4 will
prove to be hour wise but month foolish.

\paragraph{Barriers to rigorous analysis}
Another reason I am writing this book is that, like all authors, I hope
that this book will save the world. In this case, I hope to save the
world from the not-quite-correct statistics.  The reader should have
no problem finding published examples in academic journals where a
least-squares regression was run on a data set for which least-squares
was inappropriate; it is my opinion that the stats packages facilitate
and encourage this.

Most software, including all that I discuss here, has a hump: the
point where you can't get by on what you learned in the basic tutorial
and have to sit down and read the manual in earnest. For C, that hump
will coincide with your first program; after that, it will be about as
easy for you to implement the easy methods as the methods take into account
every statistical nuance you feel is important. But for stats packages
built around a fixed set of methods, the hump typically comes right after
basic GLS regressions. Implementing an off-the-shelf GLS regression
is very easy, and implementing something a touch more complicated
is difficult.  From my casual observation,
when the {\tt regress} function gets most of the assumptions right and
is basically `good enough', many will choose to just fudge things and
go with the off-the-shelf procedure.\footnote{This is not the place for
a literature review, but this ties in with an abundance of experimental
evidence that people make time-inconsistent decisions. That is, there
are many situations where being tied to the mast can make a person
better off and less regretful than when left to his or her own devices,
even if the future is entirely certain.}

Perhaps my lazy-researchers explanation is a bit baroque, and there is
a simpler story: many people simply aren't clear on what their stats package is doing.
Just as people who eat hot dogs don't really spend any
time thinking about what went in to their food, a researcher who has
a prepackaged {\tt regress} function can quickly forget the procedure
and underlying assumptions at the function's guts.  The ad copy on the
box only makes this worse, reassuring the user that the {\tt regress}
function will have no problem handling everything, so no need to worry
about reading the manual (if there is one). Again, having a pre-packaged
procedure facilitates and even encourages data analysis which is OK
but which doesn't go through the same rigor and scrutiny that the data gathering
process went through.

I don't mean to demonize stats packages or their users. I have two
packages on the computer I am writing this on, and use them regularly.
If I
were in the business of writing a statistics package for consumption by
thousands of people I've never met, most of whom are undergrads who have
just seen a regression for the first time and have just had a new piece
of software foisted upon them, then I would probably write something
that looks very much like the average stats package on the market today.
Things such as extensibility to other languages, pleasing graphics,
and of course ease of initial use, would be very important.
\index{statistics packages!rants regarding|)}

Fortunately, that's not the sort of software this book is about.
This book is intended for people who know their stats reasonably well,
have an earnest interest in doing quantitative analysis, and have
high standards of statistical correctness for their own work. For such a
person, ease of inital use is nice, but takes a back seat to ease of
getting the details right, on this project and in the long run.

So, there you have two reasons to read this book: there is the practical
position, that this book is not about ease-of-use over all other goals,
but for this very reason it will save you time over the next several
years of your career; and there is the ethical position, that in academic
research, `good enough' is not good enough, and we need software which
will accommodate an arbitrary quantity of academic rigor.
