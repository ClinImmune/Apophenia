\chapter{Bootstrapping} \label{boot}

Bootstrapping is probably one of the most eerily descriptive names I've ever seen. 
After you've gotten everything you can out of
the data, then on top of that you can bootstrap to find the variance of all of that.

At some point, I'll say more about it here.

\section{Finding the variance of a parameter}
The bootstrapping process consists of taking artificial samples of your
data, and then calculating a statistic for each of these samples. These
draws will be both independent and identically distributed.  Typically,
your statistic is for the form $\sum (X_i)/ N$, where  $N$ is the number
of observations, and $X_i$ is a data point in the case of a mean,
$(X-\mu)^2$ for a variance, et cetera. In other words, the statistic
is often the mean of some iid process, and so the Central Limit Theorem
applies to the series of statistics that you are producing (regardless
of how ugly the underlying data may be).

The primary use of this is for hypothesis testing: we need to know the
variance of the parameter and its distribution if we are to successfully
determine whether the parameter we saw is different from zero. We
solve this problem by producing a sample of the statistics, which we
are assured by the CLT will be Normally distributed, with the variance
we will calculate below. The confidence intevals calculated here will
approach the true confidence interval for the statistic
as the sample size approaches the population size.

\paragraph{An important caveat} Bootstrapping from a sample will not fix
the errors in your sample. If your sampling technique isn't perfect---and
it isn't---then it won't capture the full variation in the data. That
means that the variances you calculate using bootstrap will be less
than the true variance. In a bind, it's all you've got, and you'll
just have to state that and go on. Bootstrapping is a generally accepted technique, and has reasonable
persuasive power.

But having smaller variances makes
it easier to reject the null hypothesis, which is what your paper is
probably trying to do, so bootstrapping works slightly in your favor,
and against parsimony and skepticism. Therefore, if there is any way
of getting information about the variance of your variable without
bootstrapping, even if that estimate overstates the variance, then use
that instead of bootstrapping. Otherwise, you dishonor your name, and
bring shame to your research group.

\subsection{Creating random samples} Assume that we have a
data vector {\tt gsl\_vector * data} and a function {\tt double
calc\_statistic(gsl\_vector *data)} which finds some statistics. The
function {\tt mean()} fits this description, as would a function to find
a single OLS coefficient.

Then the first step in bootstrapping is drawing some subset of the data,
without replacement. Drawing without replacement means that we need to
keep track of what has been drawn to date, and must check every new draw
to make sure that the new draw is not a repeat. Figure \ref{bootdraw} lists a function which
does all of this. It allocates and sets up a random number generator, then
makes a series of random draws from a uniform distribution in the range
[0, {\tt data-$>$size}]. It then checks the list of indices which have
already been drawn. If it finds a match, then it invalidates the draw
by rolling back the counter by one, while if it does not find a match then it
adds the draw to the list of draws and the index to the list of indices.

\codefig{bootdraw}{A function to draw samples from the data, for use in a bootstrap}

We can then use these draws to calculate a list of parameter values, as in Figure \ref{bootvar}.
\codefig{bootvar}{A function to find the variance of a series of bootstrap estimates of a statistic}

It is valid to use 
the standard deviation returned by this function can be used in the normal way to construct confidence
intervals and test hypotheses about the statistic. 

\comment{
\section{Finding a distribution}


This section is about generating data from a model you've written yourself.

[There's even a GSL function to do this, which I stumbled upon in the
documentation for the GSL one day---in the section on drawing histograms.]
}
